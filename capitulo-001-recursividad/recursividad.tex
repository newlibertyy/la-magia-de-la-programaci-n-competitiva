\chapter{recursividad}
\section{Descripción y Motivación}

Existen problemas que para resolverlos tenemos que ejecutar el mismo bloque de instrucciones
varias veces, esto se puede lograr con ciclos iterativos o con recursividad. Todos los algoritmos
iterativos pueden ser programados recursivamente y viceversa, aun que debemos aprender a elegir
cual es la técnica correcta a utilizar. La implementación de un algoritmo iterativo consiste en repetir
el cuerpo del bucle en cambio la implementacion de un algoritmo recursivo se basa en ejecutar repetidamente el mismo metodo.
Los principales criterios a la hora de elegir entre programar algo iterativamente o recursivamente
son: el rendimiento y la simpleza del codigo generado.
Supongamos que debemos resolver el problema de sumar los primeros n numeros, dos algoritmos que
solucionan este problema son los siguientes:

\begin{minipage}{\textwidth}
Iterativo 
\begin{lstlisting}[style=C,caption=sumaIterativa.cpp]
int sumaIterativa(int n){
    int resultado = 0;
    for(int i=1;i<=n;i++){
        resultado += i;
    }
    return resultado;
}
\end{lstlisting}
\end{minipage}

	\begin{minipage}{\textwidth}
Recursivo
\begin{lstlisting}[style=C,caption=sumaRecursiva.cpp]
int sumaRecursiva(int n){
    //Caso base
    if(n==1){
        return 1;
    }else{
        return n + sumaRecursiva(n-1);
    }
}
\end{lstlisting}
\end{minipage}

Algo muy importante a tener en cuenta en los algoritmos recursivos es el caso base, al igual que
en los algoritmos iterativos se debe saber cuando detener la ejecución en los algoritmos recursivos
necesitamos saber en donde detenernos.
En realidad los dos algoritmos que mostramos tienen una ligera
diferencia aun que dan el mismo resultado. En nuestro algoritmo iterativo sumamos desde $0$ hasta $n$ de la
siguiente manera: $0+1+2+3+...+n$, pero en el recursivo sumamos desde $n$ hasta $0$: $n+(n-1)+(n-2)...+0$.
Si quisieramos que tuvieran un comportamiento mas similar podriamos programar el algoritmo recursivo
de la siguiente manera:

\begin{minipage}{\textwidth}
\begin{lstlisting}[style=C,caption=sumaRecursiva2.cpp]
int sumaRecursiva(int actual, int n){
    //Caso base
    if(actual==n){
        return actual;
    }else{
        return actual + sumaRecursiva(actual+1,n);
    }
}
\end{lstlisting}
\end{minipage}
El caso base esta muy ligado a la manera en que hacemos la recursividad, por lo general la 
recursividad se hace disminuyendo los parametros del problema, pero no siempre es asi como 
vimos en el segundo ejemplo, al igual que podemos hacer algoritmos iterativos con el contador
ascendente o descendente y tenemos que generar la condicion de detener en base a este, en la
recursividad también lo hacemos asi.
\\Quizas el ejemplo mas claro de recursividad es factorial de $n$. Ya que la solucion de factorial
de $n$ es $n * factorial de (n-1)$, y la solución de $factorial de (n-1) es (n-1) * factorial de (n-2)$ y asi consecutivamente. Por ejemplo factorial de 5 es:
\\$f(5) = 5*f(4)$
\\$f(4) = 4*f(3)$
\\$f(3) = 3*f(2)$
\\$f(2) = 2*f(1)$
\\$f(1) = 1$
\\por lo tanto
\\$f(2) = 2*f(1) = 2*1 = 2$
\\$f(3) = 3*f(2) = 3*2 = 6$
\\$f(4) = 4*f(3) = 4*6 = 24$
\\$f(5) = 5*f(4) = 5*12 = 120$
\\Mas o menos de esa manera funciona la recursividad en código, se van guardando cada llamada al 
metodo en una cola, al retornar regresa al metodo que la llamo. asi 
\\$f(5)->f(4)->f(3)->f(2)->f(1)$
\\\begin{minipage}{\textwidth}
Iterativo
\begin{lstlisting}[style=C,caption=factorialIterativo.cpp]
int factorial(int n){
    if(n==0)return 1;
    int resultado = 1;
    for(int i=n;i>=1;i--){
        resultado*=i;
    }
    return resultado;
}
\end{lstlisting}
\end{minipage}

\begin{minipage}{\textwidth}
Recursivo
\begin{lstlisting}[style=C,caption=factorialRecursivo.cpp]
int factorial(int n){
    //Caso base
    if(n==0){
        return 1;
    }
    if(n==1){
        return 1;
    }
    return n * factorial(n-1);
}
\end{lstlisting}
\end{minipage}

Se debe tener cuidado al usar recursividad en no calcular muchas veces la misma solución, por ejemplo
con el algoritmo de fibbonacci. su formula recursiva es $f(n) = f(n-1) + f(n-2)$. Si ejecutamos por ejemplo $f(5)$ sucederia lo siguiente:
\\\imagen{height=10cm,caption=fibonacci.png}{capitulo-001-recursividad/imagenes/fibonacci.png}
\\Podemos observar que recalculamos mucho, esto se puede resolver aplicando tecnicas de DP, pero eso lo veremos en otro capitulo.

\section{Ejemplos}

Ya vimos uno de los ejemplos mas tipicos de recursividad, el del factorial, en esta sección veremos
el algoritmo de fibonacci, el algoritmo de Euclides (para hallar el máximo común divisor) y 
un algoritmo para solucionar las torres de hanoi.
\\Empecemos por el algoritmo de fibonacci, por definición la suseción de fibonacci comienza de la 
siguiente forma : 0,1,1,2,3,5,8 ... , cada elemento es la suma de sus dos anteriores. Más  formalmente:

$f(n) = f(n-1) + f(n-2)$
\\Veamos primero como seria el algoritmo de fibonacci sin hacer uso de la recursión.

\begin{minipage}{\textwidth}
\begin{lstlisting}[style=C,caption=fibonacciIterativo.cpp]
int fibonacci(int n){
    if(n==0)return 0;
    if(n==1)return 1;
    int a = 0;
    int b = 1;
    int c = a+b;
    for(int i=2;i<=n;i++){
        c = a+b;
        a = b;
        b = c;
    }
    return c;
}
\end{lstlisting}
\end{minipage}

Y ahora como seria usando recursión

\begin{minipage}{\textwidth}
\begin{lstlisting}[style=C,caption=fibonacciRecursivo.cpp]
int fibonacci(int n){
    if(n==0)return 0;
    if(n==1)return 1;
    return fibonacci(n-1) + fibonacci(n-2);
}
\end{lstlisting}
\end{minipage}

Mucho más simple, ¿no lo creen?. Ahora veamos el algoritmo de euclides

\begin{minipage}{\textwidth}
Iterativo
\begin{lstlisting}[style=C,caption=euclidesIterativo.cpp]
int euclides(int a,int b){
    int temporal = a;
    while(a>0){
        temporal = a;
        a = b%a;
        b = temporal;
    }
    return b;
}
\end{lstlisting}
\end{minipage}
\begin{minipage}{\textwidth}
Recursivo
\\\begin{lstlisting}[style=C,caption=euclidesRecursivo.cpp]
int euclides(int a,int b){
    if(b==0)return a;
    return euclides(b,a%b);
}
\end{lstlisting}
\end{minipage}

Por último mi ejemplo favorito para demostrar el potencial de la recursividad,las
torres de hannoi. Si no conoces este juego, te recomiendo que primero busques
en google ``torres de hannoi online", te saldran multiples opciones para jugarlo, es bastante simple
e interesante.

En este caso no pondre una solución iterativa puesto que no se me ocurre ninguna, excepto simulando
el comportamiento de la recursividad con una cola, (para saber más detalles al respecto, te invito
a profundizar en como funciona internamente la recursividad).
\\El caso base de esta solución consiste en tener unicamente dos piezas apiladas, saber donde estan
apiladas, hacia donde se dirigen, y el otro palo sera nuestro auxiliar.
\\La solución al caso base es muy sencilla, unicamente debemos desplazar la ficha superior a nuestro
palo auxiliar, la ficha base a nuestro palo destino y por ultimo la ficha superior a nuestro destino,
y asi logramos resolver la torre de hannoi de nuestro caso base.
\\\imagen{minipageSize=0.5\linewidth,height=6cm,caption=torre1.png}{capitulo-001-recursividad/imagenes/torre1.png}
\imagen{minipageSize=0.5\linewidth,height=6cm,caption=torre2.png}{capitulo-001-recursividad/imagenes/torre2.png}
\\\imagen{minipageSize=0.5\linewidth,height=6cm,caption=torre3.png}{capitulo-001-recursividad/imagenes/torre3.png}
\imagen{minipageSize=0.5\linewidth,height=6cm,caption=torre4.png}{capitulo-001-recursividad/imagenes/torre4.png}
\\Pero que pasaria si fueran mas de dos fichas, he aqui donde viene la recursividad sucederia algo asi
\\\imagen{minipageSize=0.5\linewidth,height=6cm,caption=torre1-2.png}{capitulo-001-recursividad/imagenes/torre1-2.png}
\imagen{minipageSize=0.5\linewidth,height=6cm,caption=torre2-2.png}{capitulo-001-recursividad/imagenes/torre2-2.png}
\\\imagen{minipageSize=0.5\linewidth,height=6cm,caption=torre3-2.png}{capitulo-001-recursividad/imagenes/torre3-2.png}
\imagen{minipageSize=0.5\linewidth,height=6cm,caption=torre4-2.png}{capitulo-001-recursividad/imagenes/torre4-2.png}
Internamente la recursion de las fichas sombreadas en rojo desde ``torre1-2.png"{} hacia ``torre2-2.png"{} funcionarian de la siguiente manera:
\\\imagen{minipageSize=0.5\linewidth,height=6cm,caption=torre1-3.png}{capitulo-001-recursividad/imagenes/torre1-3.png}
\imagen{minipageSize=0.5\linewidth,height=6cm,caption=torre2-3.png}{capitulo-001-recursividad/imagenes/torre2-3.png}
\\\imagen{minipageSize=0.5\linewidth,height=6cm,caption=torre3-3.png}{capitulo-001-recursividad/imagenes/torre3-3.png}
\imagen{minipageSize=0.5\linewidth,height=6cm,caption=torre4-3.png}{capitulo-001-recursividad/imagenes/torre4-3.png}
\\Y asi sucesivamente (recursivamente). La solución recursiva de la torre de hannoi consiste en llevar la parte superior (todas las piezas menos la base) hacia el palo auxiliar, mover la base al palo destino y finalmente mover la parte superior al palo destino. Cuando la parte superior es de mas de una pieza, se realiza la recursión cambiando invirtiendo el palo destino y el auxiliar. En código seria asi:
\begin{minipage}{\textwidth}
Recursivo
\\\begin{lstlisting}[style=C,caption=hannoi.cpp]
void Hanoi(int disco, char origen, char intermedio, char destino){

  if(disco == 1){
    //caso base, solo movemos el disco a su destino
    cout << "Mover disco " << disco << " desde " << origen << " hasta " << destino << endl;
  }else{
    //movemos la parte superior al intermedio
    Hanoi(disco-1, origen,destino,intermedio);
    cout << "Mover disco " << disco << " desde " << origen << " hasta " << destino << endl;
    //movemos la parte superior al destino
    Hanoi(disco-1,intermedio,origen,destino);
  }
}

int main(){
  int discos;
  cout << "Ingrese la cantidad de discos: " << endl;
  cin >> discos;
  Hanoi(discos, 'A', 'B', 'C');

  system("pause");
}
\end{lstlisting}
\end{minipage}
