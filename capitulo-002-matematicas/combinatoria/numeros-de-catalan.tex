\subsection{Números de catalán}
Los números de catalán son una secuencia de números naturales definidos como
\\$C_{n}={\frac {1}{n+1}}{2n \choose n}={\frac {(2n)!}{(n+1)!\,n!}}\qquad {\mbox{ con }}n\geq 0.$
como era de esperarse estos también pueden ser  calculados con el triangulo de pascal
\\\imagen{minipageSize=0.5\linewidth,height=6cm,caption=catalan-pascal.png}{capitulo-002-matematicas/combinatoria/imagenes/catalan-pascal.png}
\\y su fórmula con números combinatorios es $C_{n}={2n \choose n}-{2n \choose n-1}\quad {\mbox{ con }}n\geq 1.$
\subsubsection{Aplicaciones de los números de catalán}
\begin{itemize}
  \item son el número de expresiones que tienen n pares de paréntesis correctamente colocados, para n=3 tenemos ((()))	()(())	()()()	(())()	(()())
  \item son el número de formas de  partir un polígono convexo de n+2 lados en triángulos para n=2 tenemos
  \\\imagen{minipageSize=0.5\linewidth,height=6cm,caption=poligono-catalan.png}{capitulo-002-matematicas/combinatoria/imagenes/poligono-catalan.png}
  \item número de árboles binarios que se pueden construir que tenga n+1 hojas en los que cada nodo tiene 0 ó 2 hijos, para n=2 tenemos
  \\\imagen{minipageSize=0.5\linewidth,height=6cm,caption=arbol-catalan.png}{capitulo-002-matematicas/combinatoria/imagenes/arbol-catalan.png}
  \item número de caminos que se pueden trazar por las lineas de una cuadricula de n*n sin atravesar la diagonal, para n=2 tenemos
  \\\imagen{minipageSize=0.5\linewidth,height=6cm,caption=caminos-catalan.png}{capitulo-002-matematicas/combinatoria/imagenes/caminos-catalan.png}
\end{itemize}
