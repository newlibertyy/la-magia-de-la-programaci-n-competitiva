\subsection{Números figurados}
los números figurados, son números enteros  que son posibles representarlos como una figura geométrica, muchos de ellos tienen relación con la combinatoria
\subsubsection{Números triangulares}
Estos se pueden representar como un triangulo equilátero
\\\imagen{minipageSize=0.5\linewidth,height=6cm,caption=triangular.png}{capitulo-002-matematicas/combinatoria/imagenes/triangular.png}
\\son la suma de los primeros n números naturales y su relación con la combinatoria es la siguiente:
\\Los números triangulares se encuentran en el triangulo de pascal en la tercera fila del triangulo de pascal
\\\imagen{minipageSize=0.5\linewidth,height=6cm,caption=triangulares-pascal.png}{capitulo-002-matematicas/combinatoria/imagenes/triangulares-pascal.png}
\\y el triangulo de pascal lo podemos representar como números combinatorios de la siguiente forma:
\\\imagen{minipageSize=0.5\linewidth,height=6cm,caption=triangulo-de-pascal-combinatoria.png}{capitulo-002-matematicas/combinatoria/imagenes/triangulo-de-pascal-combinatoria.png}
\\Viendo en el triangulo de pascal podemos ver que podemos representar los números triangulares como la  $T_{n}=\binom{n+1}{2} $ o usando las propiedades de los números combinados como   $T_{n}=\binom{n+1}{n-1}$
\subsubsection{Números cuadráticos}
Los números cuadráticos se pueden representar como un cuadrado
\\\imagen{minipageSize=0.5\linewidth,height=6cm,caption=cuadraticos.png}{capitulo-002-matematicas/combinatoria/imagenes/cuadraticos.png}
\\tienen un propiedad  algo extraña pero fascinante un numero cuadrático es la suma de dos números triangulares continuos así que $n^{2} = T_{n} + T_{n-1}$
\subsubsection{Números tetraédricos}
Ahora pasamos a un espacio tridimensional, estos se representan en forma de tetraedro
\\\imagen{minipageSize=0.5\linewidth,height=6cm,caption=tetraedricos.png}{capitulo-002-matematicas/combinatoria/imagenes/tetraedricos.png}
\\un tetraedro es un poliedro de 4 caras triangulares, son la suma de lo primeros n números triangulares y la 4 fila de triangulo de pascal
\\\imagen{minipageSize=0.5\linewidth,height=6cm,caption=tetraedricos-pascal.png}{capitulo-002-matematicas/combinatoria/imagenes/tetraedricos-pascal.png}
\\en el triangulo de pascal podemos ver que podemos representar los números tetraédricos como la  $Tr_{n}=\binom{n+2}{3} $ o usando las propiedades de los números combinados como   $Tr_{n}=\binom{n+2}{n-1}$
