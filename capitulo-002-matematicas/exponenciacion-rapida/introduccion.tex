\subsection{Introducción}
La potenciación rápida es un algoritmo para calcular la potencia enésima de cualquier estructura donde este definida la multiplicación y el algoritmo es el siguiente:
\begin{itemize}
\item hacemos el resultado igual a la unidad, si es impar multiplicamos el resultado por nuestra base  osea $resultado = resultado * base$
\item sin importar si es impar o no hacemos la $base =  base * base$
\item hacemos la división entera de nuestro numero  potencia por 2 $potencia=potencia/2$  y seguimos así hasta que el  número al cual estamos elevando sea 0, estamos utilizando la propiedad de la potenciación que dice $\left ( 2^{n} \right )^m=2^{mn}$
\end{itemize}

\subsubsection{Ejemplo con un números enteros:}

$2^{7} = 2* \left ( 2^{2}\right )^{3}$
\\$2^{7} = 2*2^{2} *\left (2^{4} \right )^{1}$

\subsubsection{Algoritmo general}
\begin{minipage}{\textwidth}
\begin{lstlisting}[style=C,caption=operadorPotencia]
Estructura operator^(const int& n) const
{
    Extructurra resultado(), base = *this;
    int potencia = n;
    while (potencia) {
        if (potencia & 1)//comprueba si pontencia es impar
            resultado = resultado * base;
        potencia = potencia >> 1; //es lo mismo que potencia=potenncia/2;
        base = base * base;
    }
    return resultado;
}
\end{lstlisting}
\end{minipage}
