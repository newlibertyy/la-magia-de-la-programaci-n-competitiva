\subsection{Introducción}
La potenciación rápida es un algoritmo para calcular la potencia enésima de cualquier estructura donde este definida la multiplicación y el algoritmo es el siguiente:
\begin{itemize}
\item hacemos el resultado igual a la unidad, si el exponente es impar multiplicamos el resultado por nuestra base  osea $resultado = resultado * base$
\item hacemos la $base =  base * base$
\item tomamos la parte entera de dividir nuestro exponente por 2 $exponente=exponente/2$  y seguimos así hasta que nuestro exponente sea 0, estamos utilizando la propiedad de la potenciación que dice $\left ( 2^{n} \right )^m=2^{mn}$
\end{itemize}

\subsubsection{Ejemplo con un números enteros:}

\begin{table}[htbp]
\begin{center}
\begin{tabular}{|l|l|l|}
\hline
resultado & base & exponente \\
\hline \hline
1 &	2 &	13 \\ \hline
2 &	4 &	6 \\ \hline
2	& 16 & 3 \\ \hline
32 &	256 &	1 \\ \hline
8192 &	65536 &	0 \\ \hline
\end{tabular}
\caption{potenciacion rápida con números enteros.}
\label{tabla:ejemplo}
\end{center}
\end{table}
$2^{13}=2*\left ( 2^{2} \right )^{6}$
\\$2^{13}=2*\left ( 2^{4} \right )^3$
\\$2^{13}=2*2^{4}\left ( 2^{8} \right )^1$
\\$2^{13}=2*2^{4}*2^{8} \left ( 2^{16} \right )^0$


\subsubsection{Algoritmo general}
\begin{minipage}{\textwidth}
\begin{lstlisting}[style=C,caption=operadorPotencia]
Estructura operator^(const int& n) const
{
    Extructurra resultado(), base = *this;
    int exponente = n;
    while (exponente) {
        if (exponente & 1)//comprueba si exponente es impar
            resultado = resultado * base;
        exponente = exponente >> 1; //es lo mismo que exponente=exponente/2;
        base = base * base;
    }
    return resultado;
}
\end{lstlisting}
\end{minipage}
