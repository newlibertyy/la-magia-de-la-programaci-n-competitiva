\subsection{Matrices}
Otra de las aplicaciones del algoritmo general de potenciación, es el poder elevar una matriz a un numero entero positivo,
esta operación solo es posible si la matriz es una matriz cuadrada
\\\imagen{minipageSize=0.5\linewidth,height=6cm,caption=multiplicacion-de-matrices.png}{capitulo-002-matematicas/potenciacion-rapida/imagenes/multiplicacion-de-matrices.png}
\\En muchos de los ejercicios donde se aplica lo anterior nos piden que demos un resultado modulo algún numero  es por eso
que en el código se puede apreciar que sacamos el modulo después de hacer el producto punto entre una fila una columna.
\\
\begin{minipage}{\textwidth}
\begin{lstlisting}[style=C,caption=PotenciacionMatrices.cpp]
struct Matrix {
    int v[100][100];
    int row, col;
    Matrix(int n, int m, int a = 0) {
        memset(v, 0, sizeof(v));
        row = n, col = m;
        for(int i = 0; i < row && i < col; i++)
            v[i][i] = a;
    }
    Matrix operator*(const Matrix& x) const {
        Matrix resultado(row, x.col);
        for(int i = 0; i < row; i++) {
            for(int k = 0; k < col; k++) {
                if (v[i][k])
                for(int j = 0; j < x.col; j++) {
                    resultado.v[i][j] += v[i][k] * x.v[k][j],
                    resultado.v[i][j] %= mod;
                }
            }
        }
        return resultado;
    }
    Matrix operator^(const int& n) const {
        Matrix resultado(row, col, 1), base = *this;
        int exponente = n;
        while(exponente) {
            if(exponente&1)	resultado = resultado * base;
            exponente = exponente>>1, base = base * base;
        }
        return resultado;
    }
};
\end{lstlisting}
\end{minipage}

\subsubsection{Cantidad de rutas que se pueden tomar con P pasajes}
Una de las aplicaciones de la potenciación de matrices es encontrar la cantidad de rutas que puedo
tomar en un grafo con P pasajes para llegar de un lugar a otro.
Vamos a explicar el este problema con el siguiente grafo
\\\imagen{minipageSize=0.5\linewidth,height=6cm,caption=grafo-potenciacion.png}{capitulo-002-matematicas/potenciacion-rapida/imagenes/grafo-potenciacion.png}
\\Creamos una matriz donde las filas son el nodo en el que me encuentro y la columnas el nodo al que quiero ir la intersección entre una fila y una columna es la cantidad de caminos directos que hay del nodo de la fila al nodo de la columna.
\\\imagen{minipageSize=0.5\linewidth,height=6cm,caption=matriz-de-grafo-potenciacion.png}{capitulo-002-matematicas/potenciacion-rapida/imagenes/matriz-de-grafo-potenciacion.png}
\\Esta propiedad se basa en el principio multiplicativo que vimos anteriormente, si solo queremos saber de cuantas formas podemos ir de un nodo inicial $I$ a un nodo destino $D$ solo tenemos que consultar la matriz en la posición $[I][D]$ pero si queremos saber de cuantas formas podemos ir de $I$ a $D$ con 2 pasajes multiplicamos las posibilidades de ir de $I$ a un nodo de transito y de ese nodo de transito a nuestro destino D y sumamos el resultado de todos los posibles nodos auxiliares esta es la misma operación que hacer un producto punto $fila * columna$, si en nuestro grafo queremos ir del nodo 1 al nodo 3 en 2 pasos solo tenemos que hacer lo siguiente:
$I=1$,$D=3$, $matrix[1][1]*matrix[1][3]+matrix[1][2]*matrix[2][3]+matrix[1][3]*matrix[3][3]+matrix[1][4]*matrix[4][3]+matrix[1][5]*matrix[5][3]$
si queremos hallar la cantidad de formas que podemos ir de cualquier $I$ a cualquier $D$ tenemos que hacer la multiplicación  de la matriz por ella misma y cada posición de la matriz tendrá la cantidad de formas que se puede llegar de cualquier nodo $I$ a cualquier nodo $D$ con 2 pasajes.
Si queremos en hallar la cantidad de formas en que se puede ir de un nodo $I$ a un nodo $D$ en 3 pasajes elevamos la matriz a la 3 ya si sucesivamente dependiendo de lo que necesitemos.
\\Uno de los ejercicios que se resuelve de esta manera es teletransport que lo puedes ver en \cite{Teletransport:Online}
