\subsection{Floyd’s Cycle-Finding}
Es una técnica para detectar ciclos dentro de una formula con modulo o resto de la división, se hace con la analogía de la liebre y la tortuga, el algoritmo consiste en encontrar 2 cosas.
\\$mu= cantidad de numeros que hay antes de que inicie el ciclo$
\\$lambda =cantidad de elementos que tiene el ciclo$
