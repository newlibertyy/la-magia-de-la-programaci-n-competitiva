\subsection{Floyd’s Cycle-Finding}
Es una técnica para detectar un ciclo dentro de una secuencia creada con un función f(x) de la forma g(x)mod M
donde el primer valor se da y el resto de elementos  se generaría de esta manera{x0, x1=f(x0), x2=f(x1),  …},
si tenemos la función f(x)=3x+5 mod 12 con x0=3 tendríamos {3, 2,11,2,11,2} Nuestro objetivo es encontrar 2 valores
\\$mu = cantidad de numeros que hay antes de que inicie el ciclo$
\\$lambda = cantidad de elementos que tiene el ciclo$
\\En este caso mu es 1 y lambda es 2
\\El algoritmo de detección de ciclo se hace con la analogía de la liebre y la tortuga y tiene 3 pasos
\subsubsection{primer paso:}
iniciamos la  tortuga en f(x0) y la libre en f(f(x0)) la libre se mueve 2 veces avanzamos la tortuga f(tortuga) y la libre f(f(liebre)) hasta que los 2 punteros coincidan
\subsubsection{paso 2:}
iniciamos mu=0 hacemos la liebre igual a nuestro inicio y empezamos iterar los 2 punteros paso a paso sumando le 1 a mu hasta que coincidan.
\subsubsection{paso 3:}
estando los dos punteros en el mismo lugar iniciamos lambda=1 y libre=f(liebre)
e iteramos solo con la liebre hasta que vuelva a coincidir con la tortuga sumándole 1 a lambda por cada iteración.
\begin{minipage}{\textwidth}
\begin{lstlisting}[style=C,caption=floydCycleFinding.cpp]
ii floydCycleFinding(int x0)
{
    // paso 1:
    int tortuga = f(x0), liebre = f(f(x0));
    while (tortuga != liebre) {
        tortuga = f(tortuga);
        liebre = f(f(liebre));
    }
    // paso 2:
    int mu = 0;
    liebre = x0;
    while (tortuga != liebre) {
        tortuga = f(tortuga);
        liebre = f(liebre);
        mu++;
    }
    // paso 3:
    int lambda = 1;
    liebre = f(tortuga);
    while (tortuga != liebre) {
        liebre = f(liebre);
        lambda++;
    }
    return ii(mu, lambda);
}
\end{lstlisting}
\end{minipage}
