\subsection{numeros primos}
Los números primos son todos  aquellos que son divisibles unicamente por si mismos y por uno en el conjunto del los números naturales.

\subsubsection{Criba de Eratostenes}
Es un algoritmo para hallar todos los números primos desde 1 hasta un numero n y consiste en hacer una cuadricula con
con los números y coger el primer primo elevarlo al cuadrado y tachar todos los números que a partir de su cuadrado sean
 múltiplos de este coger el siguiente numero  primo elevarlo al cuadrado y repetir el proceso cuando el numero se pase de n terminamos
y los números primos son lo que no hemos tachado.
\\Hallemos los números primos hasta el 100
\\tachamos el 1 y iniciamos con el 2 2x2=4 y a partir de ahí tachamos todos los múltiplos de 2
\\\imagen{minipageSize=0.5\linewidth,height=6cm,caption=criba-multiplos-de-2.png}{capitulo-002-matematicas/teoria-de-numeros/imagenes/criba-multiplos-de-2.png}
\\ahora el siguiente número sin tachar es el 3, 3x3=9 y a partir de ahí todos los múltiplos de 3
\\\imagen{minipageSize=0.5\linewidth,height=6cm,caption=criba-multiplos-de-2y3.png}{capitulo-002-matematicas/teoria-de-numeros/imagenes/criba-multiplos-de-2y3.png}
\\el siguiente numero sin tachar es el 5, 5x5=25 y a partir de ahí tachamos todos los múltiplos de 5,
\\\imagen{minipageSize=0.5\linewidth,height=6cm,caption=criba-multiplos-de-2-3y5.png}{capitulo-002-matematicas/teoria-de-numeros/imagenes/criba-multiplos-de-2-3y5.png}
\\el siguiente 7, 7x7=49 y a partir de ahí tachamos todos los múltiplos del 7
\\\imagen{minipageSize=0.5\linewidth,height=6cm,caption=criba-multiplos-de-2-3-5y7.png}{capitulo-002-matematicas/teoria-de-numeros/imagenes/criba-multiplos-de-2-3-5y7.png}
\\y terminamos por que el siguiente numero es 11 y 11x11= 121 que se pasa de nuestro rango, los números primos son los que no han sido marcados.
\begin{lstlisting}[style=C,caption=criba.cpp]
const int MAXN = 100;
bool criba[MAXN + 5];
vector<int> primos;
void construir_criba()
{
    memset(criba, false, sizeof(criba));
    criba[0] = criba[1] = true;
    for (int i = 2; i * i <= MAXN; i++) {
        //Coger el proximo que no este marcado
        if (!criba[i]) {
            for (int j = i * i; j <= MAXN; j += i) {
                //Marcar sus multiplos
                criba[j] = true;
            }
        }
    }
    for (int i = 2; i <= MAXN; ++i) {
        if (!criba[i])
            primos.push_back(i);
    }
}
\end{lstlisting}
Este código fue sacado y editado de las presentaciones de la universidad EAFIT ver \cite{SemilleroProgramacion:Online}

\subsubsection{Comprobar si un número es primo}
Para saber si un un numero $p$ es primo, si $p$ es menor  que nuestro $n$ retornamos la negación de la criba, si no comprobamos si este es divisible por alguno de los primos que tenemos, si alguno lo divide significa que no es primo, esto solo  funciona si $p$ es menor que el ultimo primo que tengamos al cuadrado.
\begin{minipage}{\textwidth}
\begin{lstlisting}[style=C,caption=criba.cpp]
bool esPrimo(long long N)
{
    if (N < criba.size())
        return !criba[N];

    for (int i = 0; i < (int)primos.size(); i++)
        if (N % primos[i] == 0)
            retu\begin{lstlisting}[style=C,caption=criba.cpp]

rn false;
    return true;
}
\end{lstlisting}
\end{minipage}
