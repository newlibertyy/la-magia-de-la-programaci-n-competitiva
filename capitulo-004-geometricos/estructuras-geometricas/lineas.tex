\subsection{Lineas}
Una linea es un elemento matemático que tiene infinitos puntos, una sola dimensión y va en ambos sentidos,
\\\imagen{minipageSize=0.5\linewidth,height=6cm,caption=linea.png}{capitulo-004-geometricos/estructuras-geometricas/imagenes/linea.png}
\\recomendamos usar
la forma $ax+by+c=0$ y no $y=mx+b$ por que la primera tiene la capacidad de representar lineas verticales.
\begin{lstlisting}[style=C]
struct linea { double a, b, c; };
\end{lstlisting}
\subsubsection{Operaciones con lineas}
\begin{itemize}
  \item hallar una linea con 2 puntos
  \\
  \begin{lstlisting}[style=C]
  void CrearLinea(punto p1, punto p2, linea &l) {
  if (fabs(p1.x - p2.x) < EPS) {
  //Si las x son iguales es una linea vertical
    l.a = 1.0;
    l.b = 0.0;
    l.c = -p1.x;
  } else {
    l.a = -(double)(p1.y - p2.y) / (p1.x - p2.x);
    l.b = 1.0;
    l.c = -(double)(l.a * p1.x) - p1.y;
  }
  }
  \end{lstlisting}
  \item saber si dos lineas son paralelas
  \\
  \begin{lstlisting}[style=C]
  bool sonParalelas(linea l1, linea l2) {
  return (fabs(l1.a-l2.a) < EPS) && (fabs(l1.b-l2.b) < EPS); }
  \end{lstlisting}
  \item saber si 2 lineas son iguales
  \\
  \begin{lstlisting}[style=C]
  bool sonIguales(line l1, line l2) {
  return sonParalelas(l1 ,l2) && (fabs(l1.c - l2.c) < EPS); }
  \end{lstlisting}
  \item intersección entre 2 lineas
  \\
  \begin{lstlisting}[style=C]
  bool interseccion(linea l1, linea l2, punto& p)
  {
      if (sonParalelas(l1, l2))
          //Si son paralelas la lineas no se interceptan
          return false;
      //Resolvemos el sistema de ecuaciones con dos incognitas para hallar la x
      p.x = (l2.b * l1.c - l1.b * l2.c) / (l2.a * l1.b - l1.a * l2.b);
      /* Es posible que una de nuestras rectas sea una recta vertical asi que no  podremos remplazar en ella nuestra x para hallar la y */
      if (fabs(l1.b) > EPS)
          p.y = -(l1.a * p.x + l1.c);
      else
          p.y = -(l2.a * p.x + l2.c);
      return true;
  }
  \end{lstlisting}
\end{itemize}
