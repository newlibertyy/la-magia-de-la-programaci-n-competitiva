\subsection{lineas}
\begin{lstlisting}[style=C]
struct line { double a, b, c; };
// a way to represent a line
\end{lstlisting}
\subsubsection{hallar una recta con 2 puntos}
\begin{lstlisting}[style=C]
// the answer is stored in the third parameter (pass by reference)
void pointsToLine(point p1, point p2, line &l) {
if (fabs(p1.x - p2.x) < EPS) {
// vertical line is fine
l.a = 1.0;
l.b = 0.0;
l.c = -p1.x;
// default values
} else {
l.a = -(double)(p1.y - p2.y) / (p1.x - p2.x);
l.b = 1.0;
// IMPORTANT: we fix the value of b to 1.0
l.c = -(double)(l.a * p1.x) - p1.y;
} }
\end{lstlisting}
\subsubsection{saber si dos lineas son paralelas}
\begin{lstlisting}[style=C]
bool areParallel(line l1, line l2) {
// check coefficients a & b
return (fabs(l1.a-l2.a) < EPS) && (fabs(l1.b-l2.b) < EPS); }
\end{lstlisting}
\subsubsection{saber si 2 lineas son iguales}
\begin{lstlisting}[style=C]
bool areSame(line l1, line l2) {
// also check coefficient c
return areParallel(l1 ,l2) && (fabs(l1.c - l2.c) < EPS); }
\end{lstlisting}
\subsubsection{saber si dos lineas son paralelas}
\begin{lstlisting}[style=C]
bool areParallel(line l1, line l2) {
// check coefficients a & b
return (fabs(l1.a-l2.a) < EPS) && (fabs(l1.b-l2.b) < EPS); }
\end{lstlisting}
\subsubsection{saber si 2 lineas son iguales}
\begin{lstlisting}[style=C]
bool areSame(line l1, line l2) {
// also check coefficient c
return areParallel(l1 ,l2) && (fabs(l1.c - l2.c) < EPS); }
\end{lstlisting}
\subsubsection{intersección entre 2 lineas}
