\subsection{vectores}
Un vector es un segmento de linea que tiene magnitud y dirección, los vectores son representados parecido a como se
representa un punto con dos coordenadas $x, y$ donde con eso ya tenemos la magnitud y dirección del vector en posición estándar.
\begin{lstlisting}[style=C]
struct vec { double x, y;
vec(double _x, double _y) : x(_x), y(_y) {} };
\end{lstlisting}
Si tenemos un vector que no esta en posición estándar tenemos 2 puntos $cola$ y $cabeza$ donde para trasformarlo a posición estándar
solo tenemos que restar la cola con la cabeza.
\begin{lstlisting}[style=C]
vec vecAEstandar(punto cola, punto cabeza) {
return vec(cabeza.x - cola.x, cabez.y - cola.y); }
\end{lstlisting}
\subsubsection{Operaciones con vectores}
\begin{itemize}
  \item Escalar
  \\
  Es tener un vector con una magnitud igual a la que tenia multiplicado por un numero real positivo s con la misma dirección.
  \begin{lstlisting}[style=C]
  vec escalar(vec v, double s) {
    return vec(v.x * s, v.y * s);
  }
  \end{lstlisting}
  \item Cuadrado de la magnitud
  \\
  Como un vector es un segmento de linea su magnitud se puede hallar con la formula de la distancia euclídea, si
  no sacamos la raíz tenemos la magnitud al cuadrado
  \begin{lstlisting}[style=C]
  double cuadradoMagnitud(vec v) { return v.x * v.x + v.y * v.y; }
  \end{lstlisting}
  \item Producto punto
  \\
  El producto punto es una operación entre vectores donde el resultado es un escalar
  \begin{lstlisting}[style=C]
  double ProductoPunto(vec a, vec b) { return (a.x * b.x + a.y * b.y); }
  \end{lstlisting}
  \item Producto cruz
  \\
  Normal mente el producto cruz entre 2 vectores nos da otro vector, pero a nosotros solo nos interesa la magnitud por sus
  aplicaciones al plano 2d como el área del paralelogramo formado por 2 vectores la magnitud del producto cruz la podemos aya de la
  siguiente manera
  \begin{lstlisting}[style=C]
  double cross(vec a, vec b) { return a.x * b.y - a.y * b.x; }
  \end{lstlisting}
\end{itemize}
