\chapter{Estructura de datos}
\section{Descripción y Motivación}

Una estructura de datos es la manera en la cual se organiza la información, por esta razon es posible que este capítulo sea el que más uses en tu vida cotidiana como programador.
\\Comencemos imaginando dos bibliotecas, la primera es muy estricta con sus reglas y todas las personas que leen un libro, deben regresarlo a su ubicación. En cambio la segunda biblioteca no tiene un orden, los libros estan regados por todas partes y las personas que los utilizan los dejan tirados donde sea. A primera vista pareciera que la segunda biblioteca no sirve para nada, pero en realidad si tu solo deseas ir a leer cualquier cosa y luego no tener que preocuparte de donde dejar el libro la segunda biblioteca seria ideal. A lo que quiero llegar es que hay distintas formas de ordenar la información, y algunas sirven para mejorar el desempeño en algunas areas sacrificando otras, no existe una estructura perfecta que haga bien todo al mismo tiempo.
\\Las principales operaciones sobre las estructuras son:
\begin{itemize}
    \item Insertar
    \item Buscar
    \item Borrar
    \item Actualizar
\end{itemize}
Conocer las principales estructuras de datos y entender muy bien el problema al que nos enfrentemos seran la clave para idear una solución optima.

\section{Complejidad}
No es la intención de este libro dar una explicación detallada de lo que es la complejidad de algoritmos, solo daremos una descripcion por encima de la notación big $O$. Esta notación nos dice cuantas ejecuciones realizaria un algoritmo en el peor de los casos, por ejemplo si tenemos que buscar un libro dentro de la biblioteca desordenada, la complejidad seria big $O(n)$ siendo n la cantidad de libros, ya que en el peor de los casos tendriamos que buscar uno por uno hasta el último libro.

\section{Estructuras de datos lineales}

Una estructura de datos es considerada lineal si todos sus elementos estan organizados en linea, por ejemplo en un arreglo de izquierda a derecha.
\\En la mayoria de lenguajes de programación podemos distinguir entre arreglos estaticos y arreglos dinamicos, a los arreglos estaticos les definimos un tamaño y es inalterable. 
\\\begin{minipage}{\textwidth}
\begin{lstlisting}[style=C,caption=arregloEstatico.cpp]
int main(){
    string palabras[] = {"hola","adios","tres"};
    cout<<palabras[2]<<endl;
}
\end{lstlisting}
\end{minipage}
\\Los arreglos comienzan con el indice 0 siendo palabras[0] = ``hola", palabras[1] = ``adios"{} y palabras[2]=``tres".
\subsection{Arreglos dinámicos}
Los arreglos estaticos son muy utiles cuando sabemos exactamente el tamaño de elementos que usaremos, su complejidad en las diferentes operaciones es:
\begin{itemize}
    \item Insertar/Actualizar $O(1)$ si conocemos la casilla donde insertaremos o actualizaremos, si no $O(n)$
    \item Buscar   $O(1)$ (cuando conocemos el indice), si no $O(n)$
    \item Borrar   $O(1)$ o $O(n)$ esta es una operacion complicada, ya que al borrar un elemento dejamos el espacio vacio, y lo más tipico seria correr todos los elementos de la derecha a la izquierda
\end{itemize}
Para entender un poco más esto imaginemos una estanteria de libros, donde solo caben 10 libros. Esta vacia y podemos empezar a meter libros donde queramos, pero si no tenemos un orden a la hora de ponerlos cuando esta más llena nos tomara más tiempo encontrar un espacio vacio, en cambio si vamos metiendo en orden siempre sabremos donde meter el proximo. La operación de buscar seria similar a agarrar el libro de la estanteria, si sabemos exactamente donde esta solo debemos tomarlo y ya, si no empezar a mirar uno por uno hasta encontrar el que buscamos, la operación de borrar es muy simple si solo quitamos el libro, pero hay dos cosas que podrian complicarla, la primera seria saber que libro quitaremos y la segunda si queremos que no quede el espacio vacio, pues nos tocaria correr todos los libros de la derecha hacia la izquierda para llenar el agujero. La operación de actualizar sera como una mezcla entre borrar e insertar.
\\Pero no nos asustemos, para usos prácticos es muy simple, solo usaremos arreglos estaticos para guardar información que recorreremos completa a menudo, por ejemplo si tenemos muchos amigos y a todos les queremos dar regalos:
\\\begin{minipage}{\textwidth}
\begin{lstlisting}[style=C,caption=arregloAmigos.cpp]
int main(){
    string amigos[5] = {"ana","brian","cesar","daniel","eliana"};
    string regalos[3] = {"abrazo","reloj","perfume"};

    for(int i=0;i<5;i++){
        for(int j=0;j<3;j++){
            cout<<"le regalo un "<<regalos[j]<<" a "<<amigos[i]<<endl;
        }
    }
}
\end{lstlisting}
\end{minipage}
\subsection{Arreglos esáticos}
Los arreglos dinamicos son iguales a los estaticos, excepto por que pueden agrandarse todo lo que quieran (siempre que lo soporte la RAM), otra gran diferencia es que ya traen por defecto la implementación de inserción y eliminación, esta estructura no permite huecos, por lo que su complejidad es la siguiente:
\begin{itemize}
    \item Insertar $O(1)$
    \item Buscar   $O(1)$ (cuando conocemos el indice), si no $O(n)$
    \item Borrar   $O(n)$ 
    \item actualizar $O(1)$ (cuando conocemos el indice), si no $O(n)$
\end{itemize}
Como podemos observar, sus complejidades son muy efectivas, y por eso son muy usadas en la mayoria de las ocasiones, de hecho casi cualquier problema que requiera estructura de datos se puede solucionar aplicando esta estructura, solo que obviamente no siempre es la solución óptima. Supongamos una base de datos que solo usara arreglos, seria muy lenta y poco práctica.
Un ejemplo de uso de arreglo dinámico es el siguiente:
\\\begin{minipage}{\textwidth}
\begin{lstlisting}[style=C,caption=arregloDinamicoAmigos.cpp]
int main(){
    vector<string> amigos;
    vector<string> regalos;
    string amigo,regalo;
    cout<<"ingrese todos sus amigos, uno por uno , si ya termino ingrese 0"<<endl;
    while(cin>>amigo){
        if(amigo=="0")break;
        amigos.push_back(amigo);
    }
    cout<<"ingrese todos los regalos, uno por uno , si ya termino ingrese 0"<<endl;
    while(cin>>regalo){
        if(regalo=="0")break;
        regalos.push_back(regalo);
    }

    for(int i=0;i<amigos.size();i++){
        for(int j=0;j<regalos.size();j++){
            cout<<"le regalo un "<<regalos[j]<<" a "<<amigos[i]<<endl;
        }
    }
}
\end{lstlisting}
\end{minipage}
Generalmente la unica manera de conocer el indice del elemento que estamos buscando, es que queramos recorrer el arreglo, como lo hemos hecho en los ejemplos. Asi que en la mayoria de ocasiones cuando buscamos un único elemento la complejidad es de $O(n)$, pero podemos mejorar esto, ordenando el arreglo. Como en el ejemplo de la biblioteca tener la información ordenada nos permite encontrar las cosas más rapidamente, pero sacrificamos otras cosas a cambio. Como ya lo mencionamos en estructuras de datos no hay nada perfecto para todo, tenemos dos opciones. La primera es ordenar el arreglo antes de hacer la consulta, la otra es siempre tenerlo ordenado.
Ordenar un arreglo no es una tarea fácil, por suerte la mayoria de lenguajes de programación nos provee herramientas para hacer esto, los mejores algoritmos de ordenamiento genericos tienen una complejidad de $O(n\log{}n)$, y buscar un elemento en un arreglo ordenado nos toma $O(\log{}n)$ por medio de busqueda binaria, la busqueda binaria funciona parandonos en la mitad, decidiendo si el elemento que buscamos se encuentra hacia la derecha o hacia la izquierda (lo sabemos por que estan ordenados) y repitiendo el proceso.
\imagen{minipageSize=1\linewidth,height=3cm,caption=busquedaBinaria.png}{capitulo-005-estructuras/imagenes/busquedaBinaria.png}
\\Por ejemplo si tenemos un directorio de teléfonos y estamos buscando el número de ``Sofia'', si nos paramos en la mitad del directorio encontraremos quizas las palabras que inician en ``M'', sabemos que el número que buscamos se encuentra hacia la derecha del directorio por que la ``S'' es mayor a la ``M'' asi que de una sola busqueda ya descartamos la mitad de las opciones, luego repetimos el proceso parandonos en la mitad del directorio que nos queda y esta vez nos paramos en la letra ``S'', la palabra ``Sofia'' se encuentra en esta letra asi que nos ahorramos recorrer una por una desde la ``A'' hasta la ``S'' para encontrar la pagina que buscabamos.
\\Es ineficiente ordenar un arreglo para hacer una única búsqueda, pero se vuelve efectivo a partir de una cantidad, vamos a calcular en que momento se vuelve efectivo: $S$ busquedas en un arreglo desordenado tiene una complejidad de $O(S\times{}n)$ y $S$ busquedas en un arreglo ordenado tiene una complejidad de $O(n\log{}n + S\log{}n)$. Si igualamos y despejamos $S$, obtenemos $S =\frac{nlog(n)}{n-log(n)}$ por lo tanto si nuestra cantidad de busquedas es mayor a $S$, vale la pena ordenar el arreglo antes de realizarlas.
\\Algunas implementaciones especiales que son las pilas y las colas, estas estructuras no suelen recorrersen, en cambio se usan para ingresar elementos y retirarlos con una complejidad de $O(1)$, comencemos por las colas. Funcionan igual que una cola en un restaurante, las nuevas personas que van llegando se hacen al fondo, y deben esperarsen a que atiendan a todas las que habian llegado antes que ella. Al contrario las pilas funcionan al revés, imaginemos una pila de platos, se van lavando los que estan más arriba y el ultimo que se lava es el de más abajo, si llega un nuevo plato se pone en la cima y se lava de primero.
\\\imagen{minipageSize=0.5\linewidth,height=6cm,caption=cola.png}{capitulo-005-estructuras/imagenes/cola.png}
\imagen{minipageSize=0.5\linewidth,height=6cm,caption=pila.png}{capitulo-005-estructuras/imagenes/pila.png}
Estos códigos de uso de colas y pilas fueron tomados de cplusplus y se pueden encontrar en esta url \url{http://www.cplusplus.com/reference/}.
\\\begin{minipage}{\textwidth}
\begin{lstlisting}[style=C,caption=cola.cpp]
int main(){
    queue<int> myqueue;
    int myint;
    cout << "Please enter some integers (enter 0 to end):\n";
    do {
        cin >> myint;
        myqueue.push (myint);
    } while (myint);
    cout << "myqueue contains: ";
    while (!myqueue.empty())
    {
        cout << ' ' << myqueue.front();
        myqueue.pop();
    }
    cout << '\n';
    return 0;
}

\end{lstlisting}
\end{minipage}
\\\begin{minipage}{\textwidth}
\begin{lstlisting}[style=C,caption=pila.cpp]
int main(){
    std::stack<int> mystack;

    for (int i=0; i<5; ++i) mystack.push(i);

    std::cout << "Popping out elements...";
    while (!mystack.empty())
    {
        std::cout << ' ' << mystack.top();
        mystack.pop();
    }
    std::cout << '\n';

    return 0;
}
\end{lstlisting}
\end{minipage}

\section{Estructuras de datos no lineales}

A veces las estructuras de datos lineales no son lo suficientemente eficientes para el enfoque de nuestro problema, para estos casos es probable que requiramos una estructura de datos no lineales.
\subsection{Árbol binario balanceado}
Un árbol binario balanceado es aquel que la altura de los hijos de cualquier nodo difieren en maximo 1. Los conjuntos y los mapas son codificados con esta estructura, entenderla y programarla es algo tedioso, pero los principales lenguajes de programación ya la traen implementada por defecto, si deseas conocer más acerca de esta estructura puedes consultarla por su nombre en español o la documentación en ingles como ``Balanced Binary Search Tree'', con una complejidad en todas sus operaciones de $O(log(n))$
\subsection{Conjuntos}
Los conjuntos son muy útiles cuando se quiere preguntar si un elemento existe en el conjunto, si usaramos una estructura lineal nos tomaria $O(n)$ saber si el elemento existe. Si se intenta insertar un elemento repetido no pasa nada.
\\\begin{minipage}{\textwidth}
\begin{lstlisting}[style=C,caption=conjunto.cpp]
int main(){
    set<string> palabrasFavoritas;
    string palabra;
    cout<<"inserte una palabra a tus favoritas, escriba 0 para terminar"<<endl;
    cin>>palabra;
    while(palabra!="0"){
        palabrasFavoritas.insert(palabra);
        cout<<"inserte una palabra a tus favoritas, escriba 0 para terminar"<<endl;
        cin>>palabra;
    }
    cout<<"pregunte por una palabra para saber si esta entre tus favoritas, escriba 0 para terminar"<<endl;
    string pregunta;
    cin>>pregunta;
    while(pregunta!="0"){
        if(palabrasFavoritas.count(pregunta)){
            cout<<"esta entre las favoritas"<<endl;
        }else{
            cout<<"no esta entre las favoritas"<<endl;
        }
        cout<<"pregunte por una palabra para saber si esta entre tus favoritas, escriba 0 para terminar"<<endl;
        cin>>pregunta;
    }
}
\end{lstlisting}
\end{minipage}
Existen otras aplicaciones para los conjuntos, ya que la información en estos esta siempre ordenada, se puede simular una estructura lineal con complejidad $O(log(n))$ en todas sus operaciones, recuerdan la comparación que hicimos antes sobre $S$ busquedas ordenando un arreglo, al tener esta otra manera de ordenar la información se vuelve aun mas complicado decidir que estructura de datos nos conviene más, pero como norma general seria asi:
\\\begin{itemize}
\item Si suelen hacersen muchas operaciones de insersión y casi ninguna de busqueda conviene más un arreglo dinámico sin ordenar nunca.
\item Si suelen hacersen muchas operaciones de insersión, seguidas de muchas busquedas conviene más un arreglo dinámico ordenandolo antes de iniciar la serie de busquedas.
\item Si suelen hacerse operaciones de insersión y de busquedas uniformemente, conviene más un conjunto.
\end{itemize}
Este es un claro ejemplo de por que conocer las distintas estructuras de datos nos permite optimizar nuestra solución.
\subsection{Mapas}
Los mapas son similares a los conjuntos, la única diferencia es que permiten guardar una relación entre clave$->$valor, la clave suele ser un string o un entero, pero dependiendo del lenguaje de programación puede ser de cualquier tipo de dato sobreescribiendo el operador menor que $<$, el valor puede ser cualquier tipo de dato sin ningún impedimento.
Por ejemplo si deseamos tener una registro con todos los animales y la cantidad que hemos encontrado de estos, constantemente iremos descubriendo nuevos animales y repitiendo los que ya habiamos encontrado. Si aplicamos las estructuras de datos que conociamos tendriamos que usar un arreglo dinámico con objetos que contengan el nombre del animal y la cantidad. Pero con el mapa simplemente podemos dar como clave el nombre del animal y como valor la cantidad, asi todas las operaciones tendrian complejidad de $log(n)$. Al igual que en el ejemplo de los sets, hay situaciones donde conviene más el uso de una lista, o una lista y ordenara antes que usar mapa, pero en la mayoria de aplicaciones las insersiones y busquedas tienen un comportamiento uniforme asi que conviene más el uso de mapas en la mayoria de casos.
\\\begin{minipage}{\textwidth}
\begin{lstlisting}[style=C,caption=conjunto.cpp]
int main(){
    set<string> palabrasFavoritas;
    string palabra;
    cout<<"inserte una palabra a tus favoritas, escriba 0 para terminar"<<endl;
    cin>>palabra;
    while(palabra!="0"){
        palabrasFavoritas.insert(palabra);
        cout<<"inserte una palabra a tus favoritas, escriba 0 para terminar"<<endl;
        cin>>palabra;
    }
    cout<<"pregunte por una palabra para saber si esta entre tus favoritas, escriba 0 para terminar"<<endl;
    string pregunta;
    cin>>pregunta;
    while(pregunta!="0"){
        if(palabrasFavoritas.count(pregunta)){
            cout<<"esta entre las favoritas"<<endl;
        }else{
            cout<<"no esta entre las favoritas"<<endl;
        }
        cout<<"pregunte por una palabra para saber si esta entre tus favoritas, escriba 0 para terminar"<<endl;
        cin>>pregunta;
    }
}
\end{lstlisting}
\end{minipage}
\\\subsection{Conjuntos disjuntos}
Conocida en ingles como (Union-Find Disjoint Sets) es una estructura optimizada para tener varios conjuntos y poder 	ejecutar algunas operaciones casi en tiempo lineal $\approx O(1)$, en realidad la complejidad de $M$ operaciones en esta estructura tiene una complejidad de $M*\alpha(n)$ donde $n$ es la cantidad de elementos en todos los conjuntos, y $\alpha(n)$ es la función inversa de ackerman, la función de ackerman crece muy rapido y como efecto su función inversa crece excecivamente lento. Por esto se puede considerar $\alpha(n)$ como una constante y las $M$ operaciones contarian con una complejidad $\approx O(M)$.
\\Las operaciones son:
\\\begin{itemize}
\item $Consultar(elemento)$: retorna el conjunto al que pertenece $elemento$.
\item $Evaluar(elemento1,elemento2)$: evalua si $elemento1$ esta en el mismo conjunto que $elemento2$.
\item $Unir(elemento1,elemento2)$: une el conjunto que contiene a $elemento1$ con el conjunto de $elemento2$.
\end{itemize}
Para lograr esta eficiencia, esta estructura reune todos los elementos en un árbol, de manera que el ancestro del arbol es el conjunto al que pertenecen, cuando un elemento $e_1$ se encuentra por debajo del segundo nivel del árbol (siendo el primer nivel el ancestro y el segundo nivel sus hijos directos), y se ejecuta $Consultar(e_1)$ esto tomara algunas ejecuciones hasta encontrar su ancestro, pero recursivamente iremos estableciendo al ancestro como padre directo del elemento $e_1$ y de todos sus padres.
\\\imagen{minipageSize=1\linewidth,height=13cm,caption=conjuntosDisjuntosConsultar.png}{capitulo-005-estructuras/imagenes/conjuntosDisjuntosConsultar.png}
Si queremos unir los conjuntos de dos elementos $elemento1$ y $elemento2$, lo único que debemos hacer es consultar el ancestro de ambos elementos: $Consultar(elemento1)$ y $Consultar(elemento2)$, si son distintos (pertenecen a diferente conjunto) asignamos a uno de los dos ancestros como padre del otro.
\\\imagen{minipageSize=1\linewidth,height=13cm,caption=conjuntosDisjuntosUnir.png}{capitulo-005-estructuras/imagenes/conjuntosDisjuntosUnir.png}
Para evaluar si dos elementos pertenecen al mismo conjunto unicamente debemos comparar los resultados entre $Consultar(elemento1)$ y $consultar(elemento2)$.
\\Podemos observar que el metodo $consultar$ actualiza el árbol cada que se ejecuta haciendo que el elemento y sus padres esten a solo un nodo de distancia del ancestro, y tanto $Evaluar$, como $Unir$ hacen uso de $Consultar$, es por eso que conforme se van haciendo ejecuciones, el arbol va manteniendo su tamaño reducido y sus operaciones tienden a ser $O(1)$.
\\\begin{minipage}{\textwidth}
\begin{lstlisting}[style=C,caption=unionFind.cpp]
#include <bits/stdc++.h>
using namespace std;

//1000 es el limite de elementos, puede modificarse
vector<int> pset(1000); //padre del elemento i
void inicializarConjuntos(int N) {
    pset.assign(N, 0);
    for (int i = 0; i < N; i++){
        pset[i] = i;
    }
}
int consultar(int i) {
    if(pset[i]==i){
        return i; //si ya es el ancestro lo retorna.
    }else{
        pset[i] = consultar(pset[i]); //si no es el ancestro, hace que su papa sea su ancestro y lo retorna
        return pset[i];
    }
}
bool evaluar(int i, int j) {
    return consultar(i) == consultar(j); //si tienen el mismo ancestro retorna true, si no false.
}

void unionSet(int i, int j) {
    if (!evaluar(i, j)) { //si no son el mismo conjunto, consulta el ancestro de i y de j, luego hace que el padre del ancestro de i sea el ancestro de j
        pset[consultar(i)] = consultar(j);
    }
}

int main() {
  printf("asumimos 5 elementos en 5 conjuntos diferentes al empezar\n");
  inicializarConjuntos(5); // create 5 disjoint sets
  unionSet(0, 1);
  unionSet(0, 2);
  unionSet(3, 1);
  printf("consultar(A) = %d\n", consultar(0));
  printf("consultar(B) = %d\n", consultar(1));
  printf("consultar(C) = %d\n", consultar(2));
  printf("consultar(D) = %d\n", consultar(3));
  printf("consultar(E) = %d\n", consultar(4));
  printf("evaluar(A, E) = %d\n", evaluar(0, 4));
  printf("evaluar(A, B) = %d\n", evaluar(0, 1));

  return 0;
}
\end{lstlisting}
\end{minipage}
\\Este código fue tomado y modificado de \cite{disjointSet:Online}. Todo el capítulo fue inspirado en \cite{CompetitiveProgramming3}.