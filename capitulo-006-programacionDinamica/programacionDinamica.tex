\chapter{Programación Dinamica}
\section{Descripción y Motivación}

La programación dinámica es quizás uno de los temas más complejos de tratar, por que más que teoría es casi un paradigma de programación.
\\Estudiaremos las principales y más utiles técnicas usadas en programación dinámica.
\\Pero ¿qué es la programación dinámica?, en general es aplicar técnicas para resolver problemas de optimización, maximización, minimización o conteo.


\section{Memorización}
La primera técnica que estudiaremos es la memorización, esta es muy util en algoritmos recursivos ya que evita que recalculemos desde una simple operacion hasta una rama completa de iteraciones. Esto se entiende mejor con un ejemplo, recordemos el algoritmo recursivo de fibbonacci.