\chapter{Programación Dinamica}
\section{Descripción y Motivación}

Muchos problemas de la vida real pueden ser resueltos con grafos, los grafos son conjuntos de nodos unidos por aristas, a estas aristas se les puede dar pesos u otras abstracciones necesarias, por ejemplo el mapa de una ciudad, los sitios importantes serian los nodos y las carreteras que los unen las aristas. Estas aristas podrian incluir la información de la distancia entre los nodos.
\\\imagen{minipageSize=1\linewidth,height=6cm,caption=grafoEjemplo.png}{capitulo-007-grafos/imagenes/grafoEjemplo.png}
Podemos observar como el nodo $0$ que representa al ``Laguito'' esta conectado con el nodo 1 que representa la ``Tienda'' y se encuentra a 8 unidades de distancia, las conexiones pueden ser unidireccionales o bidireccionales, en este caso podemos ir desde el ``Laguito'' a la ``Tienda'' pero no regresar. Si bien podemos ir desde la ``Biblioteca'' hasta el ``Parque'' y regresar, son dos conexiones unidireccionales separadas. Una conexión bidireccional es una sola que puede recorrerse en ambas direcciones, como una carretera de doble via.
Los grafos pueden ser representados de varias formas, las más comunes en programación son las siguientes:
\\\begin{minipage}{\textwidth}
\begin{itemize}
\item 
Matriz de adyacencia: La matriz de adyacencia consiste en un arreglo $a$ de dos dimenciones en el que $a[i][j]$ representa la conexión entre el nodo $i$ con el nodo $j$.
\\\imagen{minipageSize=1\linewidth,height=7cm,caption=matrizDeAdyacencia.png}{capitulo-007-grafos/imagenes/matrizDeAdyacencia.png}
\item
Lista de adyacencia: Consiste en tener un arreglo unidimensional por cada nodo, cada elemento de este arreglo debe contener el nodo al cual establece la conexión, también puede contener la información de la conexión.
\\\imagen{minipageSize=1\linewidth,height=8cm,caption=listaEnlazada.png}{capitulo-007-grafos/imagenes/listaEnlazada.png}
\item
Lista de aristas: Consiste en tener una unica lista unidimensional por cada conexión, cada elemento de este arreglo debe contener el nodo de origen y el nodo destino de la conexión también puede contener la información de la conexión 
\\\imagen{minipageSize=1\linewidth,height=2cm,caption=listaDeAristas.png}{capitulo-007-grafos/imagenes/listaDeAristas.png}
\end{itemize}
\end{minipage}
Cada imagen representa el mismo grafo en sus distintas formas. Que forma se elige para cual problema dependera de de que operaciones se realizaran sobre los mismos y la memoria dispuesta a ocupar, por ejemplo la matriz de adyacencia ocupa más memoria pero acceder a una conexión es instantanea pero encontrar una conexión sin conocerla para recorrer el grafo puede tomar hasta $n$ ejecuciones, en cambio la lista enlazada puede tomar hasta $e$ ejecuciones encontrar una conexión en particular, pero encontrar la primera conexión para recorrer el grafo es instantaneo.
Un buen simulador de grafos con sus distintas representaciones se puede encontrar aqui \cite{graphStructure:Online}
\section{Recorrer un grafo}
Es muy común tener que recorrer un grafo, por ejemplo si queremos saber si es posible ir desde un nodo a otro por alguna ruta.
\subsection{busqueda en profundidad}
Conocida como dfs en ingles (deep first search). A partir del nodo origen se visita su primer hijo y toda su profundidad antes de visitar sus demas hijos. Si llega a un nodo que ya ha sido visitado no lo revisita.
\\\imagen{minipageSize=1\linewidth,height=6cm,caption=dfs.png}{capitulo-007-grafos/imagenes/dfs.png}
En este caso iniciamos la busqueda desde el nodo 0, el orden de las visitas esta en letra verde.
\subsection{busqueda en anchura}
Conocida como bfs en ingles (Breadth first search). A partir del nodo origen se visitan todos sus hijos siendo estos la primera capa, luego se visitan todos sus hijos de la primera capa siendo estos la segunda capa y asi sucesivamente.
\\\imagen{minipageSize=1\linewidth,height=6cm,caption=bfs.png}{capitulo-007-grafos/imagenes/bfs.png}
Un buen simulador de recorrido de grafos en sus diferentes formas se puede encontrar aqui \cite{graphTraversal:Online}
\section{Camino más corto desde una fuente}
Para hallar el camino más corto desde una fuente usaremos el algoritmo de Dijkstra, Consiste en mantener una lista $l$ con las distancias desde el nodo origen $s$, al inicio se llena con las distancias desde el nodo $s$ hasta sus hijos, después se toma el elemento que contenga la distancia más pequeña dentro de la lista $l$ y se elimina de la lista,este elemento contiene la distancia $d_1$ que es la más pequeña hasta el nodo $n_1$, hasta el momento es obvio que ir desde $s$ hasta $n_1$ directamente es el camino más corto con una distancia de $d_1$. Ahora se añade a la lista las distancias de todos los destinos alcanzables desde $n_1$ sumada a $d_1$, despues se vuelve a buscar en la lista $l$ el elemento con la distancia $d_2$ mas pequeña, esto nos garantiza que siempre que saquemos el elemento $e_n$ de la lista $l$ contendra la distancia más corta alcanzable desde $s$ hasta $n_n$ a menos que este nodo ya lo hallamos retirado antes de la lista $l$. En resumen mantenemos una lista actualizada con todas las distancias desde $s$ a todos los nodos alcanzables por todos los caminos posibles, y la más pequeña distancia es con certeza el camino más corto desde $s$ hasta ese nodo.