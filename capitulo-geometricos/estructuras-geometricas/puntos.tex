\subsection{Puntos}
Un punto es una estructura matemática que no tiene dimensión, solo describe  una posición en el espacio. Pueden estar en el espacio
1d sobre una recta, 2d un plano … nd.
Sobre los puntos se pueden hacer varias operaciones que veremos mas adelante, la representación de un punto solo es un conjunto de
coordenadas que describen su posición, para una dimensión tendríamos un numero $x$, para dos dimensiones 2 números $x,y$
para 3 dimensiones $x,y,z$ y para n dimensiones tendríamos n números.
Estas son algunas de las formas de implementar en 2d un punto.
\begin{itemize}
\item Punto de enteros
\\
	\begin{lstlisting}[style=C]
	struct punto { int x, y;
	punto() { x = y = 0; }
	punto(int _x, int _y) : x(_x), y(_y) {} };
	\end{lstlisting}
	\item Punto de reales
	\\
	\begin{lstlisting}[style=C]
	struct punto { double x, y;
	punto() { x = y = 0.0; }
	point(double _x, double _y) : x(_x), y(_y) {} };
	\end{lstlisting}
\end{itemize}
\subsubsection{Operaciones con puntos}
\begin{itemize}
	\item Comparación
	 \\
	 Como algunos números son imposibles de representar en forma decimal por una computadora, las maquinas muchas veces aproximan el
	 resultado y esto da lugar inprecisiones  por ejemplo el numero $\frac{1}{3}$ no se puede representar en su totalidad por que tiene un número
	 de decimales infinitos, así que cuando estamos haciendo una comparación tenemos que comparar que  el valor absoluto de la resta de
	 2 valores es menor que $\varepsilon$, $\varepsilon$ es un numero muy pequeño casi cero se define normalmente como  1e-9.
	 \\
	 \begin{lstlisting}[style=C]
	 struct punto { double x, y;
	 punto() { x = y = 0.0; }
	 punto(double _x, double _y) : x(_x), y(_y) {}
	 bool operator == (punto otro) const {
	 return (fabs(x - otro.x) < EPS && (fabs(y - otro.y) < EPS));}};
	 \end{lstlisting}
	 \item Ordenamiento
	 \\
	 ordenar los puntos es muy importante en el caso de que estemos buscando optimizar la busqueda de cierto punto en un arreglo, para
	 que c++ pueda ordenar un arreglo la estructura debe tener definido el operador $<$ vamos a comprar por la coordenada $x$ y en caso
	 de empate compararemos la ordenada $y$
	 \begin{lstlisting}[style=C]
	 struct punto { double x, y;
	 punto() { x = y = 0.0; }
	 punto(double _x, double _y) : x(_x), y(_y) {}
	 bool operator < (punto otro) const {
	 if (fabs(x - otro.x) > EPS) return x < otro.x;
	 return y < otro.y; } };
	 sort(P.begin(), P.end()); //ordenar existiendo el vector P
	 \end{lstlisting}
	 \item Distancia euclídea
	 \\
	 C++ tiene ya una función implementada para hallar la hipotenusa de un triangulo de rectángulo y es hypot y la usamos como muestra la imagen 3.1
	 \\\imagen{minipageSize=0.5\linewidth,height=6cm,caption=triangulo-cartesiano.png}{capitulo-geometricos/estructuras-geometricas/imagenes/triangulo-cartesiano.png}
	 \begin{lstlisting}[style=C]
	 double dist(punto p1, punto p2) {
	 return hypot(p1.x - p2.x, p1.y - p2.y);}
	 \end{lstlisting}
\end{itemize}
