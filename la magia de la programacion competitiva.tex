\documentclass[11pt,a4paper]{book}
\usepackage[utf8]{inputenc}
\usepackage[spanish, es-lcroman]{babel}
\usepackage{amsmath}
\usepackage{amsfonts}
\usepackage{amssymb}
\usepackage{graphicx}
\usepackage{color}
\usepackage{enumerate}
\usepackage{listings}
\usepackage[font=small,labelfont=bf]{caption}
\usepackage{xparse}
\usepackage{pgfkeys}
\usepackage{keyval,xparse}% http://ctan.org/pkg/{keyval,xparse}
\usepackage{cite}
\definecolor{gray97}{gray}{.97}
\definecolor{gray75}{gray}{.75}
\definecolor{gray45}{gray}{.45}

\lstset{ frame=Ltb,
     framerule=0pt,
     aboveskip=0.5cm,
     framextopmargin=3pt,
     framexbottommargin=3pt,
     framexleftmargin=0.4cm,
     framesep=0pt,
     rulesep=.4pt,
     backgroundcolor=\color{gray97},
     rulesepcolor=\color{black},
     %
     stringstyle=\ttfamily,
     showstringspaces = false,
     basicstyle=\small\ttfamily,
     commentstyle=\color{gray45},
     keywordstyle=\bfseries,
     %
     numbers=left,
     numbersep=15pt,
     numberstyle=\tiny,
     numberfirstline = false,
     breaklines=true,
   }

% minimizar fragmentado de listados
\lstnewenvironment{listing}[1][]
   {\lstset{#1}\pagebreak[0]}{\pagebreak[0]}

\lstdefinestyle{consola}
   {basicstyle=\scriptsize\bf\ttfamily,
    backgroundcolor=\color{gray75},
   }

\lstdefinestyle{C}
   {language=C++,}

\usepackage[left=2cm,right=2cm,top=2cm,bottom=2cm]{geometry}


\author{Comunidad new liberty}
\title{La magia de la programación competitiva}

\begin{document}
%creamos el metodo para hacer imagenes
\makeatletter
% ========= KEY DEFINITIONS =========
\define@key{imagen}{height}{\def\imagen@height{#1}}
\define@key{imagen}{url}{\def\imagen@url{#1}}
\define@key{imagen}{minipageSize}{\def\imagen@minipageSize{#1}}
\define@key{imagen}{caption}{\def\imagen@caption{#1}}
\DeclareDocumentCommand{\imagen}{m m}{%
  \begingroup%
  % ========= KEY DEFAULTS + new ones =========
  \setkeys{imagen}{height={2cm},url={ahorita miro},minipageSize={0.3\linewidth},caption={ahorita miro segundo},#1}%
	\begin{minipage}{\imagen@minipageSize}
	\includegraphics[height=\imagen@height]{#2}
	\captionsetup{justification=centering}
	\captionof{figure}{\imagen@caption}
	\end{minipage}
\endgroup%
}
\makeatother

\maketitle
\tableofcontents
\cleardoublepage
\addcontentsline{toc}{chapter}{Lista de figuras}
\listoffigures
\cleardoublepage
\addcontentsline{toc}{chapter}{Lista de tablas}
\listoftables
\cleardoublepage

\chapter{recursividad}
\section{Descripción y Motivación}

Existen problemas que para resolverlos tenemos que ejecutar el mismo bloque de instrucciones
varias veces, esto se puede lograr con ciclos iterativos o con recursividad. Todos los algoritmos
iterativos pueden ser programados recursivamente y viceversa, aun que debemos aprender a elegir
cual es la técnica correcta a utilizar. La implementación de un algoritmo iterativo consiste en repetir
el cuerpo del bucle en cambio la implementacion de un algoritmo recursivo se basa en ejecutar repetidamente el mismo metodo.
Los principales criterios a la hora de elegir entre programar algo iterativamente o recursivamente
son: el rendimiento y la simpleza del codigo generado.
Supongamos que debemos resolver el problema de sumar los primeros n numeros, dos algoritmos que
solucionan este problema son los siguientes:

\begin{minipage}{\textwidth}
Iterativo 
\begin{lstlisting}[style=C,caption=sumaIterativa.cpp]
int sumaIterativa(int n){
    int resultado = 0;
    for(int i=1;i<=n;i++){
        resultado += i;
    }
    return resultado;
}
\end{lstlisting}
\end{minipage}

	\begin{minipage}{\textwidth}
Recursivo
\begin{lstlisting}[style=C,caption=sumaRecursiva.cpp]
int sumaRecursiva(int n){
    //Caso base
    if(n==1){
        return 1;
    }else{
        return n + sumaRecursiva(n-1);
    }
}
\end{lstlisting}
\end{minipage}

Algo muy importante a tener en cuenta en los algoritmos recursivos es el caso base, al igual que
en los algoritmos iterativos se debe saber cuando detener la ejecución en los algoritmos recursivos
necesitamos saber en donde detenernos.
En realidad los dos algoritmos que mostramos tienen una ligera
diferencia aun que dan el mismo resultado. En nuestro algoritmo iterativo sumamos desde $0$ hasta $n$ de la
siguiente manera: $0+1+2+3+...+n$, pero en el recursivo sumamos desde $n$ hasta $0$: $n+(n-1)+(n-2)...+0$.
Si quisieramos que tuvieran un comportamiento mas similar podriamos programar el algoritmo recursivo
de la siguiente manera:

\begin{minipage}{\textwidth}
\begin{lstlisting}[style=C,caption=sumaRecursiva2.cpp]
int sumaRecursiva(int actual, int n){
    //Caso base
    if(actual==n){
        return actual;
    }else{
        return actual + sumaRecursiva(actual+1,n);
    }
}
\end{lstlisting}
\end{minipage}
El caso base esta muy ligado a la manera en que hacemos la recursividad, por lo general la 
recursividad se hace disminuyendo los parametros del problema, pero no siempre es asi como 
vimos en el segundo ejemplo, al igual que podemos hacer algoritmos iterativos con el contador
ascendente o descendente y tenemos que generar la condicion de detener en base a este, en la
recursividad también lo hacemos asi.
\\Quizas el ejemplo mas claro de recursividad es factorial de $n$. Ya que la solucion de factorial
de $n$ es $n * factorial de (n-1)$, y la solución de $factorial de (n-1) es (n-1) * factorial de (n-2)$ y asi consecutivamente. Por ejemplo factorial de 5 es:
\\$f(5) = 5*f(4)$
\\$f(4) = 4*f(3)$
\\$f(3) = 3*f(2)$
\\$f(2) = 2*f(1)$
\\$f(1) = 1$
\\por lo tanto
\\$f(2) = 2*f(1) = 2*1 = 2$
\\$f(3) = 3*f(2) = 3*2 = 6$
\\$f(4) = 4*f(3) = 4*6 = 24$
\\$f(5) = 5*f(4) = 5*12 = 120$
\\Mas o menos de esa manera funciona la recursividad en código, se van guardando cada llamada al 
metodo en una cola, al retornar regresa al metodo que la llamo. asi 
\\$f(5)->f(4)->f(3)->f(2)->f(1)$
\\\begin{minipage}{\textwidth}
Iterativo
\begin{lstlisting}[style=C,caption=factorialIterativo.cpp]
int factorial(int n){
    if(n==0)return 1;
    int resultado = 1;
    for(int i=n;i>=1;i--){
        resultado*=i;
    }
    return resultado;
}
\end{lstlisting}
\end{minipage}

\begin{minipage}{\textwidth}
Recursivo
\begin{lstlisting}[style=C,caption=factorialRecursivo.cpp]
int factorial(int n){
    //Caso base
    if(n==0){
        return 1;
    }
    if(n==1){
        return 1;
    }
    return n * factorial(n-1);
}
\end{lstlisting}
\end{minipage}

Se debe tener cuidado al usar recursividad en no calcular muchas veces la misma solución, por ejemplo
con el algoritmo de fibbonacci. su formula recursiva es $f(n) = f(n-1) + f(n-2)$. Si ejecutamos por ejemplo $f(5)$ sucederia lo siguiente:
\\\imagen{height=10cm,caption=fibonacci.png}{capitulo-001-recursividad/imagenes/fibonacci.png}
\\Podemos observar que recalculamos mucho, esto se puede resolver aplicando tecnicas de DP, pero eso lo veremos en otro capitulo.

\section{Ejemplos}

Ya vimos uno de los ejemplos mas tipicos de recursividad, el del factorial, en esta sección veremos
el algoritmo de fibonacci, el algoritmo de Euclides (para hallar el máximo común divisor) y 
un algoritmo para solucionar las torres de hanoi.
\\Empecemos por el algoritmo de fibonacci, por definición la suseción de fibonacci comienza de la 
siguiente forma : 0,1,1,2,3,5,8 ... , cada elemento es la suma de sus dos anteriores. Más  formalmente:

$f(n) = f(n-1) + f(n-2)$
\\Veamos primero como seria el algoritmo de fibonacci sin hacer uso de la recursión.

\begin{minipage}{\textwidth}
\begin{lstlisting}[style=C,caption=fibonacciIterativo.cpp]
int fibonacci(int n){
    if(n==0)return 0;
    if(n==1)return 1;
    int a = 0;
    int b = 1;
    int c = a+b;
    for(int i=2;i<=n;i++){
        c = a+b;
        a = b;
        b = c;
    }
    return c;
}
\end{lstlisting}
\end{minipage}

Y ahora como seria usando recursión

\begin{minipage}{\textwidth}
\begin{lstlisting}[style=C,caption=fibonacciRecursivo.cpp]
int fibonacci(int n){
    if(n==0)return 0;
    if(n==1)return 1;
    return fibonacci(n-1) + fibonacci(n-2);
}
\end{lstlisting}
\end{minipage}

Mucho más simple, ¿no lo creen?. Ahora veamos el algoritmo de euclides

\begin{minipage}{\textwidth}
Iterativo
\begin{lstlisting}[style=C,caption=euclidesIterativo.cpp]
int euclides(int a,int b){
    int temporal = a;
    while(a>0){
        temporal = a;
        a = b%a;
        b = temporal;
    }
    return b;
}
\end{lstlisting}
\end{minipage}
\begin{minipage}{\textwidth}
Recursivo
\\\begin{lstlisting}[style=C,caption=euclidesRecursivo.cpp]
int euclides(int a,int b){
    if(b==0)return a;
    return euclides(b,a%b);
}
\end{lstlisting}
\end{minipage}

Por último mi ejemplo favorito para demostrar el potencial de la recursividad,las
torres de hannoi. Si no conoces este juego, te recomiendo que primero busques
en google ``torres de hannoi online", te saldran multiples opciones para jugarlo, es bastante simple
e interesante.

En este caso no pondre una solución iterativa puesto que no se me ocurre ninguna, excepto simulando
el comportamiento de la recursividad con una cola, (para saber más detalles al respecto, te invito
a profundizar en como funciona internamente la recursividad).
\\El caso base de esta solución consiste en tener unicamente dos piezas apiladas, saber donde estan
apiladas, hacia donde se dirigen, y el otro palo sera nuestro auxiliar.
\\La solución al caso base es muy sencilla, unicamente debemos desplazar la ficha superior a nuestro
palo auxiliar, la ficha base a nuestro palo destino y por ultimo la ficha superior a nuestro destino,
y asi logramos resolver la torre de hannoi de nuestro caso base.
\\\imagen{minipageSize=0.5\linewidth,height=6cm,caption=torre1.png}{capitulo-001-recursividad/imagenes/torre1.png}
\imagen{minipageSize=0.5\linewidth,height=6cm,caption=torre2.png}{capitulo-001-recursividad/imagenes/torre2.png}
\\\imagen{minipageSize=0.5\linewidth,height=6cm,caption=torre3.png}{capitulo-001-recursividad/imagenes/torre3.png}
\imagen{minipageSize=0.5\linewidth,height=6cm,caption=torre4.png}{capitulo-001-recursividad/imagenes/torre4.png}
\\Pero que pasaria si fueran mas de dos fichas, he aqui donde viene la recursividad sucederia algo asi
\\\imagen{minipageSize=0.5\linewidth,height=6cm,caption=torre1-2.png}{capitulo-001-recursividad/imagenes/torre1-2.png}
\imagen{minipageSize=0.5\linewidth,height=6cm,caption=torre2-2.png}{capitulo-001-recursividad/imagenes/torre2-2.png}
\\\imagen{minipageSize=0.5\linewidth,height=6cm,caption=torre3-2.png}{capitulo-001-recursividad/imagenes/torre3-2.png}
\imagen{minipageSize=0.5\linewidth,height=6cm,caption=torre4-2.png}{capitulo-001-recursividad/imagenes/torre4-2.png}
Internamente la recursion de las fichas sombreadas en rojo desde ``torre1-2.png"{} hacia ``torre2-2.png"{} funcionarian de la siguiente manera:
\\\imagen{minipageSize=0.5\linewidth,height=6cm,caption=torre1-3.png}{capitulo-001-recursividad/imagenes/torre1-3.png}
\imagen{minipageSize=0.5\linewidth,height=6cm,caption=torre2-3.png}{capitulo-001-recursividad/imagenes/torre2-3.png}
\\\imagen{minipageSize=0.5\linewidth,height=6cm,caption=torre3-3.png}{capitulo-001-recursividad/imagenes/torre3-3.png}
\imagen{minipageSize=0.5\linewidth,height=6cm,caption=torre4-3.png}{capitulo-001-recursividad/imagenes/torre4-3.png}
\\Y asi sucesivamente (recursivamente). La solución recursiva de la torre de hannoi consiste en llevar la parte superior (todas las piezas menos la base) hacia el palo auxiliar, mover la base al palo destino y finalmente mover la parte superior al palo destino. Cuando la parte superior es de mas de una pieza, se realiza la recursión cambiando invirtiendo el palo destino y el auxiliar. En código seria asi:
\begin{minipage}{\textwidth}
Recursivo
\\\begin{lstlisting}[style=C,caption=hannoi.cpp]
void Hanoi(int disco, char origen, char intermedio, char destino){

  if(disco == 1){
    //caso base, solo movemos el disco a su destino
    cout << "Mover disco " << disco << " desde " << origen << " hasta " << destino << endl;
  }else{
    //movemos la parte superior al intermedio
    Hanoi(disco-1, origen,destino,intermedio);
    cout << "Mover disco " << disco << " desde " << origen << " hasta " << destino << endl;
    //movemos la parte superior al destino
    Hanoi(disco-1,intermedio,origen,destino);
  }
}

int main(){
  int discos;
  cout << "Ingrese la cantidad de discos: " << endl;
  cin >> discos;
  Hanoi(discos, 'A', 'B', 'C');

  system("pause");
}
\end{lstlisting}
\end{minipage}

\chapter{Matemáticas}
\section{sucesiones y series}
\subsection{sucesión aritmética}
las sucesiones aritméticas son aquellas que restando un elemento con su antecesor siempre da una constante se representan de la siguiente manera.
\\$ an+b$
\\donde a es la resta entre dos elementos consecutivos y b es el primer elemento
\subsection{sucesión geométrica}
las sucesiones geométricas son aquellas que el cociente de un elemento con su antecesor siempre da una constante se representan de la siguiente manera.
\\$ ar^{n-1}$
\\donde a es el primer termino y r es el cociente entre un numero y su anterior
\subsection{Serie aritmética}
Una serie aritmética es una sucesión creada con la suma de los términos de una sucesión aritmética, su formula es:
\\$a\frac{n(n+1)}{2}+nb$

\subsection{serie geométrica}
una serie geométrica es una sucesión creada con la suma del los términos de una sucesión geométrica, su formula es:
\\$a\frac{1-r^{n}}{1-r}$
\section{sumatorios}
los sumatorios son la suma de elementos de una secuencia, estas son las propiedades:
\begin{itemize}
\item la cantidad de elementos de un sumatorio es el limite superior menos el limite inferior mas la unidad
\item el sumatorio de una constante es la cantidad de elementos por la constante
\item el sumatorio es una transformación lineal o aplicación lineal y cumple con todas sus propiedades
\end{itemize}

\section{teoria de numeros}
\subsection{aritmetica modular}
\input{capitulo-002-matematicas/teoria-de-numeros/aritmetica-modular/maximo-comun-divisor-y-minimo-comun-multiplo.tex}
\input{capitulo-002-matematicas/teoria-de-numeros/aritmetica-modular/algoritmo-de-euclides-extendido.tex}
\input{capitulo-002-matematicas/teoria-de-numeros/aritmetica-modular/ecuaciones-diofanticas-lineales-de-dos.variables.tex}
\input{capitulo-002-matematicas/teoria-de-numeros/aritmetica-modular/conjunto-zn.tex}
\input{capitulo-002-matematicas/teoria-de-numeros/aritmetica-modular/propiedades-de-la-aritmetica-modular.tex}
\input{capitulo-002-matematicas/teoria-de-numeros/aritmetica-modular/inverzo-multiplicativo-en-zn.tex}


\section{combinatoria}
\subsection{principio multiplicativo}
si se quiere realizar un procedimiento de n pasos donde el primer paso puede ser hecho de $a_{1}$, el segundo paso
de $a_{2}$ y así sucesivamente hasta $a_{n}$ las formas de llevar a cabo el procedimiento son $a_{1}*a_{2} ... *a_{n}$

\subsection{numero de permutaciones de n elementos}
el numero de permutaciones es el numero de arreglos donde el orden importa, el numero de permutaciones
se calcula como $P(n,n)=n!$

\subsection{numero de permutaciones de n elementos tomados de a m}
el numero de permutaciones de n elementos tomados de a m son
$P(n,m)=\frac{n!}{\left (n-m  \right )!}$
\subsection{número de permutaciones de n elementos tomados de a m con repetición}
en este problema tenemos un suministro ilimitado de los n elementos diferentes y queremos saber de cuantas maneras podemos coger m elementos. su formula es:
$Pr(n,m)=n^{m}$

\subsection{Número de permutaciones con al menos un elemento fijo}
El número de permutaciones que tienen al menos un elemento fijo son todas las permutaciones que no son desarreglos\\
$n!-D_{n}$

\subsection{Número de permutaciones donde el primer elemento se repite a veces el segundo b veces ...}
el número de permutaciones es:
$\frac{n!}{a!b!...}$

\subsection{Número de desarreglos}
El numero de desarreglos es el numero de permutaciones que podemos hacer donde ninguno de los elementos esta en su posición inicial, se calculan con la siguiente formula recursiva.\\
$D_{n}=(n-1)(D_{n-1}+D_{n-2})$\\
casos base:
$D_{2}=1$ $D_{3}=2$
\subsection{Número de permutaciones de n elementos que dejan exactamente k elementos fijos}
El numero de permutaciones que dejan exactamente k elementos fijos es lo mismo que tachar k elementos y hacer un desarreglo con los n-k restantes. entonces la formula seria el numero de formas que podemos escoger k elementos del total multiplicado el desarreglo de n-k, siendo $s(n,k)$ el numero de arreglos con exactamente k elementos fijos tenemos:\\
$s(n,k)=c(n,k)D_{n-k}$

\subsection{número de combinaciones de n elementos tomados de a m}
el numero de combinaciones de n elementos cogidos de m son el numero de formas que podemos coger m elementos de los n sin importar su orden\\
\subsubsection{formas de calcular los números combinados o coeficientes binomiales}
formula:\\
$c(n,m)=\frac{n!}{m!(n-m)!}$\\
\\
formula recursiva:\\
casos bases $c(n,n)=c(n,0)=1$\\
de mas casos $c(n,m)=c(n-1,k)+c(n-1,k-1)$

\subsection{números figurados}
los números figurados, son números enteros  que son posibles representarlos como una figura geométrica, muchos de ellos tienen relación con la combinatoria
\subsubsection{números triangulares}
estos se pueden representar como un triangulo equilátero
\\\imagen{minipageSize=0.5\linewidth,height=6cm,caption=triangular.png}{capitulo-002-matematicas/combinatoria/imagenes/triangular.png}
\\son la suma de los primeros n números naturales y su relación con la combinatoria es la siguiente:
\\Los números triangulares se encuentran en el triangulo de pascal en la tercera fila del triangulo de pascal
\\\imagen{minipageSize=0.5\linewidth,height=6cm,caption=triangulares-pascal.png}{capitulo-002-matematicas/combinatoria/imagenes/triangulares-pascal.png}
\\y el triangulo de pascal lo podemos representar como números combinatorios de la siguiente forma:
\\\imagen{minipageSize=0.5\linewidth,height=6cm,caption=triangulo-de-pascal-combinatoria.png}{capitulo-002-matematicas/combinatoria/imagenes/triangulo-de-pascal-combinatoria.png}
\\Viendo en el triangulo de pascal podemos ver que podemos representar los números triangulares como la  $T_{n}=\binom{n+1}{2} $ o usando las propiedades de los números combinados como   $T_{n}=\binom{n+1}{n-1}$
\subsubsection{Números cuadráticos}
Los números cuadráticos se pueden representar como un cuadrado
\\\imagen{minipageSize=0.5\linewidth,height=6cm,caption=cuadraticos.png}{capitulo-002-matematicas/combinatoria/imagenes/cuadraticos.png}
\\tienen un propiedad  algo extraña pero fascinante un numero cuadrático en la suma de dos números triangulares continuos así que $n^{2} = T_{n} + T_{n-1}$

\subsection{Números de fibonacci}
Los números de fibonacci  ademas de aparecer en muchos de los patrones de la naturaleza también se pueden calcular con el triangulo de pascal
\\\imagen{minipageSize=0.5\linewidth,height=6cm,caption=fibonacci-pascal.png}{capitulo-002-matematicas/combinatoria/imagenes/fibonacci-pascal.png}
$fib(n+1)=\sum_{k=0}^{\frac{n}{2}}\binom{n-k}{k}$

\subsection{números de catalán}
los números de catalán son una secuencia de números naturales definidos como
\\$C_{n}={\frac {1}{n+1}}{2n \choose n}={\frac {(2n)!}{(n+1)!\,n!}}\qquad {\mbox{ con }}n\geq 0.$
como era de esperarse estos también pueden ser  calculados con el triangulo de pascal
\\\imagen{minipageSize=0.5\linewidth,height=6cm,caption=catalan-pascal.png}{capitulo-002-matematicas/combinatoria/imagenes/catalan-pascal.png}
\\y su fórmula con números combinatorios es $C_{n}={2n \choose n}-{2n \choose n-1}\quad {\mbox{ con }}n\geq 1.$
\subsubsection{Aplicaciones de los números de catalán}
\begin{itemize}
  \item son el número de expresiones que tienen n pares de paréntesis correctamente colocados, para n=3 tenemos ((()))	()(())	()()()	(())()	(()())
  \item son el número de formas de  partir un polígono convexo de n+2 lados en triángulos para n=2 tenemos
  \\\imagen{minipageSize=0.5\linewidth,height=6cm,caption=poligono-catalan.png}{capitulo-002-matematicas/combinatoria/imagenes/poligono-catalan.png}
  \item número de árboles binarios que se pueden construir que tenga n+1 hojas en los que cada nodo tiene 0 ó 2 hijos, para n=2 tenemos
  \\\imagen{minipageSize=0.5\linewidth,height=6cm,caption=arbol-catalan.png}{capitulo-002-matematicas/combinatoria/imagenes/arbol-catalan.png}
  \item número de caminos que se pueden trazar por las lineas de una cuadricula de n*n sin atravesar la diagonal, para n=2 tenemos
  \\\imagen{minipageSize=0.5\linewidth,height=6cm,caption=caminos-catalan.png}{capitulo-002-matematicas/combinatoria/imagenes/caminos-catalan.png}
\end{itemize}


\section{probabilidad}
\subsection{regla de Laplace}
la regla de laplace establece que la probabilidad de que ocurra un evento es la cantidad de casos favorables sobre la cantidad de casos posibles\\
$p(x)=\frac{casos\_favorables}{casos\_posibles}$

\subsection{probabilidad de intersección de sucesos}
si tenemos dos sucesos $a$ y $b$ la probabilidad de que suceda $a$ y $b$ es:
$p(a\cap b)=p(a)P(b|a)$\\

\subsection{Probabilidad de unión de sucesos}
Si tenemos dos sucesos $a$ y $b$ la probabilidad de que suceda $a$ ó $b$ es:
$p(a\cup b)=p(a)+p(b)-p(a\cap b)$

\subsection{Probabilidad condicionada}
La probabilidad condicionada es la probabilidad de que ocurra un evento $a$ sabiendo que ya ocurrió un evento $b$ y se calcula de la siguiente manera $p(a|b)=\frac{p(a\cap b)}{p(b)}$

\subsection{teorema de Bayes}
el teorema de Bayes indica una relación entre $p(a|b)$ y $p(b|a)$ y puede ser sacado de las formulas anteriores que hemos visto $p(a|b)=\frac{p(a)P(b|a)}{p(b)}$


\section{Potenciación rápida}
\subsection{Introducción}
La potenciación rápida es un algoritmo para calcular la potencia enésima de cualquier estructura donde este definida la multiplicación y el algoritmo es el siguiente:
\\\\mientras exponente sea diferente de 0 se repiten los siguientes pasos
\begin{itemize}
\item hacemos el resultado igual a la unidad, si el exponente es impar multiplicamos el resultado por nuestra base  osea $resultado = resultado * base$
\item hacemos la $base =  base * base$
\item tomamos la parte entera de dividir nuestro exponente por 2 $exponente=exponente/2$, estamos utilizando la propiedad de la potenciación que dice $\left ( 2^{n} \right )^m=2^{mn}$
\end{itemize}

\subsubsection{Ejemplo con un números enteros:}

\begin{table}[htbp]
\begin{center}
\begin{tabular}{|l|l|l|}
\hline
resultado & base & exponente \\
\hline \hline
1 &	2 &	13 \\ \hline
2 &	4 &	6 \\ \hline
2	& 16 & 3 \\ \hline
32 &	256 &	1 \\ \hline
8192 &	65536 &	0 \\ \hline
\end{tabular}
\caption{potenciacion rápida con números enteros.}
\label{tabla:ejemplo}
\end{center}
\end{table}
$2^{13}=2*\left ( 2^{2} \right )^{6}$
\\$2^{13}=2*\left ( 2^{4} \right )^3$
\\$2^{13}=2*2^{4}\left ( 2^{8} \right )^1$
\\$2^{13}=2*2^{4}*2^{8} \left ( 2^{16} \right )^0$


\subsubsection{Algoritmo general}
\begin{minipage}{\textwidth}
\begin{lstlisting}[style=C,caption=operadorPotencia]
Estructura operator^(const int& n) const
{
    Extructurra resultado(), base = *this;
    int exponente = n;
    while (exponente) {
        if (exponente & 1)//comprueba si exponente es impar
            resultado = resultado * base;
        exponente = exponente >> 1; //es lo mismo que exponente=exponente/2;
        base = base * base;
    }
    return resultado;
}
\end{lstlisting}
\end{minipage}


\section{transformaciones lineales}
una transformación lineal es una función que satisface los siguientes axiomas
\begin{itemize}
\item $f(x+y)=f(x)+f(y)$
\item $f(ax)=af(x)$ siendo a una constante
\end{itemize}


\chapter{Geometricos}
\section{formulas de geometría}
\begin{itemize}
\item $\frac{a}{sin(A)}=\frac{b}{sin(B)}=\frac{c}{sin(C)}$
\end{itemize}

\section{Estructuras geométricas}
\subsection{Puntos}
Un punto es una estructura matemática que no tiene dimensión, solo describe  una posición en el espacio. Pueden estar en el espacio
1d sobre una recta, 2d un plano … nd.
Sobre los puntos se pueden hacer varias operaciones que veremos mas adelante, la representación de un punto solo es un conjunto de
coordenadas que describen su posición, para una dimensión tendríamos un numero $x$, para dos dimensiones 2 números $x,y$
para 3 dimensiones $x,y,z$ y para n dimensiones tendríamos n números.
Estas son algunas de las formas de implementar en 2d un punto.
\begin{itemize}
\item Punto de enteros
\\
	\begin{lstlisting}[style=C]
	struct punto { int x, y;
	punto() { x = y = 0; }
	punto(int _x, int _y) : x(_x), y(_y) {} };
	\end{lstlisting}
	\item Punto de reales
	\\
	\begin{lstlisting}[style=C]
	struct punto { double x, y;
	punto() { x = y = 0.0; }
	point(double _x, double _y) : x(_x), y(_y) {} };
	\end{lstlisting}
\end{itemize}
\subsubsection{Operaciones con puntos}
\begin{itemize}
	\item Comparación
	 \\
	 Como algunos números son imposibles de representar en forma decimal por una computadora, las maquinas muchas veces aproximan el
	 resultado y esto da lugar inprecisiones  por ejemplo el numero $\frac{1}{3}$ no se puede representar en su totalidad por que tiene un número
	 de decimales infinitos, así que cuando estamos haciendo una comparación tenemos que comparar que  el valor absoluto de la resta de
	 2 valores es menor que $\varepsilon$, $\varepsilon$ es un numero muy pequeño casi cero se define normalmente como  1e-9.
	 \\
	 \begin{lstlisting}[style=C]
	 struct punto { double x, y;
	 punto() { x = y = 0.0; }
	 punto(double _x, double _y) : x(_x), y(_y) {}
	 bool operator == (punto otro) const {
	 return (fabs(x - otro.x) < EPS && (fabs(y - otro.y) < EPS));}};
	 \end{lstlisting}
	 \item Ordenamiento
	 \\
	 ordenar los puntos es muy importante en el caso de que estemos buscando optimizar la busqueda de cierto punto en un arreglo, para
	 que c++ pueda ordenar un arreglo la estructura debe tener definido el operador $<$ vamos a comprar por la coordenada $x$ y en caso
	 de empate compararemos la ordenada $y$
	 \begin{lstlisting}[style=C]
	 struct punto { double x, y;
	 punto() { x = y = 0.0; }
	 punto(double _x, double _y) : x(_x), y(_y) {}
	 bool operator < (punto otro) const {
	 if (fabs(x - otro.x) > EPS) return x < otro.x
	 return y < otro.y; } };
	 sort(P.begin(), P.end()); //ordenar existiendo el vector P
	 \end{lstlisting}
	 \item Distancia euclídea
	 \\
	 C++ tiene ya una función implementada para hallar la hipotenusa de un triangulo de rectángulo y es hypot y la usamos como muestra la imagen 3.1
	 \\\imagen{minipageSize=0.5\linewidth,height=6cm,caption=triangulo-cartesiano.png}{capitulo-004-geometricos/estructuras-geometricas/imagenes/triangulo-cartesiano.png}
	 \begin{lstlisting}[style=C]
	 double dist(punto p1, punto p2) {
	 return hypot(p1.x - p2.x, p1.y - p2.y);}
	 \end{lstlisting}
\end{itemize}

\subsection{Lineas}
\begin{lstlisting}[style=C]
struct line { double a, b, c; };
// a way to represent a line
\end{lstlisting}
\subsubsection{hallar una recta con 2 puntos}
\begin{lstlisting}[style=C]
// the answer is stored in the third parameter (pass by reference)
void pointsToLine(point p1, point p2, line &l) {
if (fabs(p1.x - p2.x) < EPS) {
// vertical line is fine
l.a = 1.0;
l.b = 0.0;
l.c = -p1.x;
// default values
} else {
l.a = -(double)(p1.y - p2.y) / (p1.x - p2.x);
l.b = 1.0;
// IMPORTANT: we fix the value of b to 1.0
l.c = -(double)(l.a * p1.x) - p1.y;
} }
\end{lstlisting}
\subsubsection{saber si dos lineas son paralelas}
\begin{lstlisting}[style=C]
bool areParallel(line l1, line l2) {
// check coefficients a & b
return (fabs(l1.a-l2.a) < EPS) && (fabs(l1.b-l2.b) < EPS); }
\end{lstlisting}
\subsubsection{saber si 2 lineas son iguales}
\begin{lstlisting}[style=C]
bool areSame(line l1, line l2) {
// also check coefficient c
return areParallel(l1 ,l2) && (fabs(l1.c - l2.c) < EPS); }
\end{lstlisting}
\subsubsection{saber si dos lineas son paralelas}
\begin{lstlisting}[style=C]
bool areParallel(line l1, line l2) {
// check coefficients a & b
return (fabs(l1.a-l2.a) < EPS) && (fabs(l1.b-l2.b) < EPS); }
\end{lstlisting}
\subsubsection{saber si 2 lineas son iguales}
\begin{lstlisting}[style=C]
bool areSame(line l1, line l2) {
// also check coefficient c
return areParallel(l1 ,l2) && (fabs(l1.c - l2.c) < EPS); }
\end{lstlisting}
\subsubsection{intersección entre 2 lineas}

\subsection{vectores}
Un vector es un segmento de linea que tiene magnitud y dirección, los vectores son representados parecido a como se
representa un punto con dos coordenadas $x, y$ donde con eso ya tenemos la magnitud y dirección del vector en posición estándar.
\begin{lstlisting}[style=C]
struct vec { double x, y;
vec(double _x, double _y) : x(_x), y(_y) {} };
\end{lstlisting}
Si tenemos un vector que no esta en posición estándar tenemos 2 puntos $cola$ y $cabeza$ donde para trasformarlo a posición estándar
solo tenemos que restar la cola con la cabeza.
\begin{lstlisting}[style=C]
vec vecAEstandar(punto cola, punto cabeza) {
return vec(cabeza.x - cola.x, cabez.y - cola.y); }
\end{lstlisting}
\subsubsection{Operaciones con vectores}
\begin{itemize}
  \item Escalar
  \\
  Es tener un vector con una magnitud igual a la que tenia multiplicado por un numero real positivo s con la misma dirección.
  \begin{lstlisting}[style=C]
  vec escalar(vec v, double s) {
    return vec(v.x * s, v.y * s);
  }
  \end{lstlisting}
  \item Cuadrado de la magnitud
  \\
  Como un vector es un segmento de linea su magnitud se puede hallar con la formula de la distancia euclídea, si
  no sacamos la raíz tenemos la magnitud al cuadrado
  \begin{lstlisting}[style=C]
  double cuadradoMagnitud(vec v) { return v.x * v.x + v.y * v.y; }
  \end{lstlisting}
  \item Producto punto
  \\
  El producto punto es una operación entre vectores donde el resultado es un escalar
  \begin{lstlisting}[style=C]
  double ProductoPunto(vec a, vec b) { return (a.x * b.x + a.y * b.y); }
  \end{lstlisting}
  \item Producto cruz
  \\
  Normal mente el producto cruz entre 2 vectores nos da otro vector, pero a nosotros solo nos interesa la magnitud por sus
  aplicaciones al plano 2d como el área del paralelogramo formado por 2 vectores. la magnitud del producto cruz la podemos aya de la
  siguiente manera
  \begin{lstlisting}[style=C]
  double cross(vec a, vec b) { return a.x * b.y - a.y * b.x; }
  \end{lstlisting}
\end{itemize}
\subsubsection{Aplicaciones de los vectores}
\begin{itemize}
  \item Área de un paralelogramo
  \\
  \\\imagen{minipageSize=0.5\linewidth,height=6cm,caption=paralelogramo.png}{capitulo-004-geometricos/estructuras-geometricas/imagenes/paralelogramo.png}
  \\
  $\left ( 2,5 \right )$ y $\left (4,1 \right )$
  \item Saber si un punto esta a la derecha o la izquierda  de una recta o esta dentro de la recta
  \\El producto punto se pude escribir también como $\sin \left ( \Theta  \right )\left | a \right |\left | b \right |$
  \\\imagen{minipageSize=0.5\linewidth,height=6cm,caption=dercha-izquierda-colineal.png}{capitulo-004-geometricos/estructuras-geometricas/imagenes/dercha-izquierda-colineal.png}
  \\
  Si el punto esta a la izquierda el seno del angulo sera positivo, si esta a la izquierda sera negativo y si es lineal sera 0.
\end{itemize}



\chapter{Estructura de datos}
\section{Descripción y Motivación}

Una estructura de datos es la manera en la cual se organiza la información, por esta razon es posible que este capítulo sea el que más uses en tu vida cotidiana como programador.
\\Comencemos imaginando dos bibliotecas, la primera es muy estricta con sus reglas y todas las personas que leen un libro, deben regresarlo a su ubicación. En cambio la segunda biblioteca no tiene un orden, los libros estan regados por todas partes y las personas que los utilizan los dejan tirados donde sea. A primera vista pareciera que la segunda biblioteca no sirve para nada, pero en realidad si tu solo deseas ir a leer cualquier cosa y luego no tener que preocuparte de donde dejar el libro la segunda biblioteca seria ideal. A lo que quiero llegar es que hay distintas formas de ordenar la información, y algunas sirven para mejorar el desempeño en algunas areas sacrificando otras, no existe una estructura perfecta que haga bien todo al mismo tiempo.
\\Las principales operaciones sobre las estructuras son:
\begin{itemize}
    \item Insertar
    \item Buscar
    \item Borrar
    \item Actualizar
\end{itemize}
Conocer las principales estructuras de datos y entender muy bien el problema al que nos enfrentemos seran la clave para idear una solución optima.

\section{Complejidad}
No es la intención de este libro dar una explicación detallada de lo que es la complejidad de algoritmos, solo daremos una descripcion por encima de la notación big $O$. Esta notación nos dice cuantas ejecuciones realizaria un algoritmo en el peor de los casos, por ejemplo si tenemos que buscar un libro dentro de la biblioteca desordenada, la complejidad seria big $O(n)$ siendo n la cantidad de libros, ya que en el peor de los casos tendriamos que buscar uno por uno hasta el último libro.

\section{Estructuras de datos lineales}

Una estructura de datos es considerada lineal si todos sus elementos estan organizados en linea, por ejemplo en un arreglo de izquierda a derecha.
\\En la mayoria de lenguajes de programación podemos distinguir entre arreglos estaticos y arreglos dinamicos, a los arreglos estaticos les definimos un tamaño y es inalterable. 
\\\begin{minipage}{\textwidth}
\begin{lstlisting}[style=C,caption=arregloEstatico.cpp]
int main(){
    string palabras[] = {"hola","adios","tres"};
    cout<<palabras[2]<<endl;
}
\end{lstlisting}
\end{minipage}
\\Los arreglos comienzan con el indice 0 siendo palabras[0] = ``hola", palabras[1] = ``adios"{} y palabras[2]=``tres".
\subsection{Arreglos dinámicos}
Los arreglos estaticos son muy utiles cuando sabemos exactamente el tamaño de elementos que usaremos, su complejidad en las diferentes operaciones es:
\begin{itemize}
    \item Insertar/Actualizar $O(1)$ si conocemos la casilla donde insertaremos o actualizaremos, si no $O(n)$
    \item Buscar   $O(1)$ (cuando conocemos el indice), si no $O(n)$
    \item Borrar   $O(1)$ o $O(n)$ esta es una operacion complicada, ya que al borrar un elemento dejamos el espacio vacio, y lo más tipico seria correr todos los elementos de la derecha a la izquierda
\end{itemize}
Para entender un poco más esto imaginemos una estanteria de libros, donde solo caben 10 libros. Esta vacia y podemos empezar a meter libros donde queramos, pero si no tenemos un orden a la hora de ponerlos cuando esta más llena nos tomara más tiempo encontrar un espacio vacio, en cambio si vamos metiendo en orden siempre sabremos donde meter el proximo. La operación de buscar seria similar a agarrar el libro de la estanteria, si sabemos exactamente donde esta solo debemos tomarlo y ya, si no empezar a mirar uno por uno hasta encontrar el que buscamos, la operación de borrar es muy simple si solo quitamos el libro, pero hay dos cosas que podrian complicarla, la primera seria saber que libro quitaremos y la segunda si queremos que no quede el espacio vacio, pues nos tocaria correr todos los libros de la derecha hacia la izquierda para llenar el agujero. La operación de actualizar sera como una mezcla entre borrar e insertar.
\\Pero no nos asustemos, para usos prácticos es muy simple, solo usaremos arreglos estaticos para guardar información que recorreremos completa a menudo, por ejemplo si tenemos muchos amigos y a todos les queremos dar regalos:
\\\begin{minipage}{\textwidth}
\begin{lstlisting}[style=C,caption=arregloAmigos.cpp]
int main(){
    string amigos[5] = {"ana","brian","cesar","daniel","eliana"};
    string regalos[3] = {"abrazo","reloj","perfume"};

    for(int i=0;i<5;i++){
        for(int j=0;j<3;j++){
            cout<<"le regalo un "<<regalos[j]<<" a "<<amigos[i]<<endl;
        }
    }
}
\end{lstlisting}
\end{minipage}
\subsection{Arreglos esáticos}
Los arreglos dinamicos son iguales a los estaticos, excepto por que pueden agrandarse todo lo que quieran (siempre que lo soporte la RAM), otra gran diferencia es que ya traen por defecto la implementación de inserción y eliminación, esta estructura no permite huecos, por lo que su complejidad es la siguiente:
\begin{itemize}
    \item Insertar $O(1)$
    \item Buscar   $O(1)$ (cuando conocemos el indice), si no $O(n)$
    \item Borrar   $O(n)$ 
    \item actualizar $O(1)$ (cuando conocemos el indice), si no $O(n)$
\end{itemize}
Como podemos observar, sus complejidades son muy efectivas, y por eso son muy usadas en la mayoria de las ocasiones, de hecho casi cualquier problema que requiera estructura de datos se puede solucionar aplicando esta estructura, solo que obviamente no siempre es la solución óptima. Supongamos una base de datos que solo usara arreglos, seria muy lenta y poco práctica.
Un ejemplo de uso de arreglo dinámico es el siguiente:
\\\begin{minipage}{\textwidth}
\begin{lstlisting}[style=C,caption=arregloDinamicoAmigos.cpp]
int main(){
    vector<string> amigos;
    vector<string> regalos;
    string amigo,regalo;
    cout<<"ingrese todos sus amigos, uno por uno , si ya termino ingrese 0"<<endl;
    while(cin>>amigo){
        if(amigo=="0")break;
        amigos.push_back(amigo);
    }
    cout<<"ingrese todos los regalos, uno por uno , si ya termino ingrese 0"<<endl;
    while(cin>>regalo){
        if(regalo=="0")break;
        regalos.push_back(regalo);
    }

    for(int i=0;i<amigos.size();i++){
        for(int j=0;j<regalos.size();j++){
            cout<<"le regalo un "<<regalos[j]<<" a "<<amigos[i]<<endl;
        }
    }
}
\end{lstlisting}
\end{minipage}
Generalmente la unica manera de conocer el indice del elemento que estamos buscando, es que queramos recorrer el arreglo, como lo hemos hecho en los ejemplos. Asi que en la mayoria de ocasiones cuando buscamos un único elemento la complejidad es de $O(n)$, pero podemos mejorar esto, ordenando el arreglo. Como en el ejemplo de la biblioteca tener la información ordenada nos permite encontrar las cosas más rapidamente, pero sacrificamos otras cosas a cambio. Como ya lo mencionamos en estructuras de datos no hay nada perfecto para todo, tenemos dos opciones. La primera es ordenar el arreglo antes de hacer la consulta, la otra es siempre tenerlo ordenado.
Ordenar un arreglo no es una tarea fácil, por suerte la mayoria de lenguajes de programación nos provee herramientas para hacer esto, los mejores algoritmos de ordenamiento genericos tienen una complejidad de $O(n\log{}n)$, y buscar un elemento en un arreglo ordenado nos toma $O(\log{}n)$ por medio de busqueda binaria, la busqueda binaria funciona parandonos en la mitad, decidiendo si el elemento que buscamos se encuentra hacia la derecha o hacia la izquierda (lo sabemos por que estan ordenados) y repitiendo el proceso.
\imagen{minipageSize=1\linewidth,height=3cm,caption=busquedaBinaria.png}{capitulo-005-estructuras/imagenes/busquedaBinaria.png}
\\Por ejemplo si tenemos un directorio de teléfonos y estamos buscando el número de ``Sofia'', si nos paramos en la mitad del directorio encontraremos quizas las palabras que inician en ``M'', sabemos que el número que buscamos se encuentra hacia la derecha del directorio por que la ``S'' es mayor a la ``M'' asi que de una sola busqueda ya descartamos la mitad de las opciones, luego repetimos el proceso parandonos en la mitad del directorio que nos queda y esta vez nos paramos en la letra ``S'', la palabra ``Sofia'' se encuentra en esta letra asi que nos ahorramos recorrer una por una desde la ``A'' hasta la ``S'' para encontrar la pagina que buscabamos.
\\Es ineficiente ordenar un arreglo para hacer una única búsqueda, pero se vuelve efectivo a partir de una cantidad, vamos a calcular en que momento se vuelve efectivo: $S$ busquedas en un arreglo desordenado tiene una complejidad de $O(S\times{}n)$ y $S$ busquedas en un arreglo ordenado tiene una complejidad de $O(n\log{}n + S\log{}n)$. Si igualamos y despejamos $S$, obtenemos $S =\frac{nlog(n)}{n-log(n)}$ por lo tanto si nuestra cantidad de busquedas es mayor a $S$, vale la pena ordenar el arreglo antes de realizarlas.
\\Algunas implementaciones especiales que son las pilas y las colas, estas estructuras no suelen recorrersen, en cambio se usan para ingresar elementos y retirarlos con una complejidad de $O(1)$, comencemos por las colas. Funcionan igual que una cola en un restaurante, las nuevas personas que van llegando se hacen al fondo, y deben esperarsen a que atiendan a todas las que habian llegado antes que ella. Al contrario las pilas funcionan al revés, imaginemos una pila de platos, se van lavando los que estan más arriba y el ultimo que se lava es el de más abajo, si llega un nuevo plato se pone en la cima y se lava de primero.
\\\imagen{minipageSize=0.5\linewidth,height=6cm,caption=cola.png}{capitulo-005-estructuras/imagenes/cola.png}
\imagen{minipageSize=0.5\linewidth,height=6cm,caption=pila.png}{capitulo-005-estructuras/imagenes/pila.png}
Estos códigos de uso de colas y pilas fueron tomados de cplusplus y se pueden encontrar en esta url \url{http://www.cplusplus.com/reference/}.
\\\begin{minipage}{\textwidth}
\begin{lstlisting}[style=C,caption=cola.cpp]
int main(){
    queue<int> myqueue;
    int myint;
    cout << "Please enter some integers (enter 0 to end):\n";
    do {
        cin >> myint;
        myqueue.push (myint);
    } while (myint);
    cout << "myqueue contains: ";
    while (!myqueue.empty())
    {
        cout << ' ' << myqueue.front();
        myqueue.pop();
    }
    cout << '\n';
    return 0;
}

\end{lstlisting}
\end{minipage}
\\\begin{minipage}{\textwidth}
\begin{lstlisting}[style=C,caption=pila.cpp]
int main(){
    std::stack<int> mystack;

    for (int i=0; i<5; ++i) mystack.push(i);

    std::cout << "Popping out elements...";
    while (!mystack.empty())
    {
        std::cout << ' ' << mystack.top();
        mystack.pop();
    }
    std::cout << '\n';

    return 0;
}
\end{lstlisting}
\end{minipage}

\section{Estructuras de datos no lineales}

A veces las estructuras de datos lineales no son lo suficientemente eficientes para el enfoque de nuestro problema, para estos casos es probable que requiramos una estructura de datos no lineales.
\subsection{Árbol binario balanceado}
Un árbol binario balanceado es aquel que la altura de los hijos de cualquier nodo difieren en maximo 1. Los conjuntos y los mapas son codificados con esta estructura, entenderla y programarla es algo tedioso, pero los principales lenguajes de programación ya la traen implementada por defecto, si deseas conocer más acerca de esta estructura puedes consultarla por su nombre en español o la documentación en ingles como ``Balanced Binary Search Tree'', con una complejidad en todas sus operaciones de $O(log(n))$
\subsection{Conjuntos}
Los conjuntos son muy útiles cuando se quiere preguntar si un elemento existe en el conjunto, si usaramos una estructura lineal nos tomaria $O(n)$ saber si el elemento existe. Si se intenta insertar un elemento repetido no pasa nada.
\\\begin{minipage}{\textwidth}
\begin{lstlisting}[style=C,caption=conjunto.cpp]
int main(){
    set<string> palabrasFavoritas;
    string palabra;
    cout<<"inserte una palabra a tus favoritas, escriba 0 para terminar"<<endl;
    cin>>palabra;
    while(palabra!="0"){
        palabrasFavoritas.insert(palabra);
        cout<<"inserte una palabra a tus favoritas, escriba 0 para terminar"<<endl;
        cin>>palabra;
    }
    cout<<"pregunte por una palabra para saber si esta entre tus favoritas, escriba 0 para terminar"<<endl;
    string pregunta;
    cin>>pregunta;
    while(pregunta!="0"){
        if(palabrasFavoritas.count(pregunta)){
            cout<<"esta entre las favoritas"<<endl;
        }else{
            cout<<"no esta entre las favoritas"<<endl;
        }
        cout<<"pregunte por una palabra para saber si esta entre tus favoritas, escriba 0 para terminar"<<endl;
        cin>>pregunta;
    }
}
\end{lstlisting}
\end{minipage}
Existen otras aplicaciones para los conjuntos, ya que la información en estos esta siempre ordenada, se puede simular una estructura lineal con complejidad $O(log(n))$ en todas sus operaciones, recuerdan la comparación que hicimos antes sobre $S$ busquedas ordenando un arreglo, al tener esta otra manera de ordenar la información se vuelve aun mas complicado decidir que estructura de datos nos conviene más, pero como norma general seria asi:
\\\begin{itemize}
\item Si suelen hacersen muchas operaciones de insersión y casi ninguna de busqueda conviene más un arreglo dinámico sin ordenar nunca.
\item Si suelen hacersen muchas operaciones de insersión, seguidas de muchas busquedas conviene más un arreglo dinámico ordenandolo antes de iniciar la serie de busquedas.
\item Si suelen hacerse operaciones de insersión y de busquedas uniformemente, conviene más un conjunto.
\end{itemize}
Este es un claro ejemplo de por que conocer las distintas estructuras de datos nos permite optimizar nuestra solución.
\subsection{Mapas}
Los mapas son similares a los conjuntos, la única diferencia es que permiten guardar una relación entre clave$->$valor, la clave suele ser un string o un entero, pero dependiendo del lenguaje de programación puede ser de cualquier tipo de dato sobreescribiendo el operador menor que $<$, el valor puede ser cualquier tipo de dato sin ningún impedimento.
Por ejemplo si deseamos tener una registro con todos los animales y la cantidad que hemos encontrado de estos, constantemente iremos descubriendo nuevos animales y repitiendo los que ya habiamos encontrado. Si aplicamos las estructuras de datos que conociamos tendriamos que usar un arreglo dinámico con objetos que contengan el nombre del animal y la cantidad. Pero con el mapa simplemente podemos dar como clave el nombre del animal y como valor la cantidad, asi todas las operaciones tendrian complejidad de $log(n)$. Al igual que en el ejemplo de los sets, hay situaciones donde conviene más el uso de una lista, o una lista y ordenara antes que usar mapa, pero en la mayoria de aplicaciones las insersiones y busquedas tienen un comportamiento uniforme asi que conviene más el uso de mapas en la mayoria de casos.
\\\begin{minipage}{\textwidth}
\begin{lstlisting}[style=C,caption=conjunto.cpp]
int main(){
    set<string> palabrasFavoritas;
    string palabra;
    cout<<"inserte una palabra a tus favoritas, escriba 0 para terminar"<<endl;
    cin>>palabra;
    while(palabra!="0"){
        palabrasFavoritas.insert(palabra);
        cout<<"inserte una palabra a tus favoritas, escriba 0 para terminar"<<endl;
        cin>>palabra;
    }
    cout<<"pregunte por una palabra para saber si esta entre tus favoritas, escriba 0 para terminar"<<endl;
    string pregunta;
    cin>>pregunta;
    while(pregunta!="0"){
        if(palabrasFavoritas.count(pregunta)){
            cout<<"esta entre las favoritas"<<endl;
        }else{
            cout<<"no esta entre las favoritas"<<endl;
        }
        cout<<"pregunte por una palabra para saber si esta entre tus favoritas, escriba 0 para terminar"<<endl;
        cin>>pregunta;
    }
}
\end{lstlisting}
\end{minipage}
\\\subsection{Conjuntos disjuntos}
Conocida en ingles como (Union-Find Disjoint Sets) es una estructura optimizada para tener varios conjuntos y poder 	ejecutar algunas operaciones casi en tiempo lineal $\approx O(1)$, en realidad la complejidad de $M$ operaciones en esta estructura tiene una complejidad de $M*\alpha(n)$ donde $n$ es la cantidad de elementos en todos los conjuntos, y $\alpha(n)$ es la función inversa de ackerman, la función de ackerman crece muy rapido y como efecto su función inversa crece excecivamente lento. Por esto se puede considerar $\alpha(n)$ como una constante y las $M$ operaciones contarian con una complejidad $\approx O(M)$.
\\Las operaciones son:
\\\begin{itemize}
\item $Consultar(elemento)$: retorna el conjunto al que pertenece $elemento$.
\item $Evaluar(elemento1,elemento2)$: evalua si $elemento1$ esta en el mismo conjunto que $elemento2$.
\item $Unir(elemento1,elemento2)$: une el conjunto que contiene a $elemento1$ con el conjunto de $elemento2$.
\end{itemize}
Para lograr esta eficiencia, esta estructura reune todos los elementos en un árbol, de manera que el ancestro del arbol es el conjunto al que pertenecen, cuando un elemento $e_1$ se encuentra por debajo del segundo nivel del árbol (siendo el primer nivel el ancestro y el segundo nivel sus hijos directos), y se ejecuta $Consultar(e_1)$ esto tomara algunas ejecuciones hasta encontrar su ancestro, pero recursivamente iremos estableciendo al ancestro como padre directo del elemento $e_1$ y de todos sus padres.
\\\imagen{minipageSize=1\linewidth,height=13cm,caption=conjuntosDisjuntosConsultar.png}{capitulo-005-estructuras/imagenes/conjuntosDisjuntosConsultar.png}
Si queremos unir los conjuntos de dos elementos $elemento1$ y $elemento2$, lo único que debemos hacer es consultar el ancestro de ambos elementos: $Consultar(elemento1)$ y $Consultar(elemento2)$, si son distintos (pertenecen a diferente conjunto) asignamos a uno de los dos ancestros como padre del otro.
\\\imagen{minipageSize=1\linewidth,height=13cm,caption=conjuntosDisjuntosUnir.png}{capitulo-005-estructuras/imagenes/conjuntosDisjuntosUnir.png}
Para evaluar si dos elementos pertenecen al mismo conjunto unicamente debemos comparar los resultados entre $Consultar(elemento1)$ y $consultar(elemento2)$.
\\Podemos observar que el metodo $consultar$ actualiza el árbol cada que se ejecuta haciendo que el elemento y sus padres esten a solo un nodo de distancia del ancestro, y tanto $Evaluar$, como $Unir$ hacen uso de $Consultar$, es por eso que conforme se van haciendo ejecuciones, el arbol va manteniendo su tamaño reducido y sus operaciones tienden a ser $O(1)$.
\\\begin{minipage}{\textwidth}
\begin{lstlisting}[style=C,caption=unionFind.cpp]
#include <bits/stdc++.h>
using namespace std;

//1000 es el limite de elementos, puede modificarse
vector<int> pset(1000); //padre del elemento i
void inicializarConjuntos(int N) {
    pset.assign(N, 0);
    for (int i = 0; i < N; i++){
        pset[i] = i;
    }
}
int consultar(int i) {
    if(pset[i]==i){
        return i; //si ya es el ancestro lo retorna.
    }else{
        pset[i] = consultar(pset[i]); //si no es el ancestro, hace que su papa sea su ancestro y lo retorna
        return pset[i];
    }
}
bool evaluar(int i, int j) {
    return consultar(i) == consultar(j); //si tienen el mismo ancestro retorna true, si no false.
}

void unionSet(int i, int j) {
    if (!evaluar(i, j)) { //si no son el mismo conjunto, consulta el ancestro de i y de j, luego hace que el padre del ancestro de i sea el ancestro de j
        pset[consultar(i)] = consultar(j);
    }
}

int main() {
  printf("asumimos 5 elementos en 5 conjuntos diferentes al empezar\n");
  inicializarConjuntos(5); // create 5 disjoint sets
  unionSet(0, 1);
  unionSet(0, 2);
  unionSet(3, 1);
  printf("consultar(A) = %d\n", consultar(0));
  printf("consultar(B) = %d\n", consultar(1));
  printf("consultar(C) = %d\n", consultar(2));
  printf("consultar(D) = %d\n", consultar(3));
  printf("consultar(E) = %d\n", consultar(4));
  printf("evaluar(A, E) = %d\n", evaluar(0, 4));
  printf("evaluar(A, B) = %d\n", evaluar(0, 1));

  return 0;
}
\end{lstlisting}
\end{minipage}
\\Este código fue tomado y modificado de \cite{disjointSet:Online}. Todo el capítulo fue inspirado en \cite{CompetitiveProgramming3}.
\\
\\
Puedes verlo en \cite{Patricio2011}. Te recomiendo leer \cite{Patricio2011, Zacarias2009, Alfonso2010b, Alfonso2010a}.
\cleardoublepage
\addcontentsline{toc}{chapter}{Bibliografía}
\bibliographystyle{acm} % estilo de la bibliografía.
\bibliography{bibliografia}
\end{document}
