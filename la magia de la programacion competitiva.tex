\documentclass[11pt,a4paper]{book}
\usepackage[utf8]{inputenc}
\usepackage[spanish, es-lcroman]{babel} 
\usepackage{amsmath}
\usepackage{amsfonts}
\usepackage{amssymb}
\usepackage{graphicx}
\usepackage{color}
\usepackage{enumerate}
\definecolor{gray97}{gray}{.97}
\definecolor{gray75}{gray}{.75}
\definecolor{gray45}{gray}{.45}
 
\usepackage{listings}
\lstset{ frame=Ltb,
     framerule=0pt,
     aboveskip=0.5cm,
     framextopmargin=3pt,
     framexbottommargin=3pt,
     framexleftmargin=0.4cm,
     framesep=0pt,
     rulesep=.4pt,
     backgroundcolor=\color{gray97},
     rulesepcolor=\color{black},
     %
     stringstyle=\ttfamily,
     showstringspaces = false,
     basicstyle=\small\ttfamily,
     commentstyle=\color{gray45},
     keywordstyle=\bfseries,
     %
     numbers=left,
     numbersep=15pt,
     numberstyle=\tiny,
     numberfirstline = false,
     breaklines=true,
   }
 
% minimizar fragmentado de listados
\lstnewenvironment{listing}[1][]
   {\lstset{#1}\pagebreak[0]}{\pagebreak[0]}
 
\lstdefinestyle{consola}
   {basicstyle=\scriptsize\bf\ttfamily,
    backgroundcolor=\color{gray75},
   }
 
\lstdefinestyle{C}
   {language=C++,}
 
\usepackage[left=2cm,right=2cm,top=2cm,bottom=2cm]{geometry}
\author{Comunidad new liberty}
\title{La magia de la programación competitiva}

\begin{document}
\maketitle
\tableofcontents
\cleardoublepage
\addcontentsline{toc}{chapter}{Lista de figuras} 
\listoffigures
\cleardoublepage
\addcontentsline{toc}{chapter}{Lista de tablas}
\listoftables
\cleardoublepage
\chapter{Matemáticas}
\section{sucesiones y series}
\subsection{sucesión aritmética}
las sucesiones aritméticas son aquellas que restando un elemento con su antecesor siempre da una constante se representan de la siguiente manera.
\\$ an+b$
\\donde a es la resta entre dos elementos consecutivos y b es el primer elemento
\subsection{sucesión geométrica}
las sucesiones geométricas son aquellas que el cociente de un elemento con su antecesor siempre da una constante se representan de la siguiente manera.
\\$ ar^{n-1}$
\\donde a es el primer termino y r es el cociente entre un numero y su anterior
\subsection{Serie aritmética}
Una serie aritmética es una sucesión creada con la suma de los términos de una sucesión aritmética, su formula es:
\\$a\frac{n(n+1)}{2}+nb$

\subsection{serie geométrica}
una serie geométrica es una sucesión creada con la suma del los términos de una sucesión geométrica, su formula es:
\\$a\frac{1-r^{n}}{1-r}$
\section{sumatorios}
los sumatorios son la suma de elementos de una secuencia, estas son las propiedades:
\begin{itemize}
\item la cantidad de elementos de un sumatorio es el limite superior menos el limite inferior mas la unidad
\item el sumatorio de una constante es la cantidad de elementos por la constante
\item el sumatorio es una transformación lineal o aplicación lineal y cumple con todas sus propiedades
\end{itemize}

\section{teoria de numeros}
\subsection{aritmetica modular}
\input{capitulo-002-matematicas/teoria-de-numeros/aritmetica-modular/maximo-comun-divisor-y-minimo-comun-multiplo.tex}
\input{capitulo-002-matematicas/teoria-de-numeros/aritmetica-modular/algoritmo-de-euclides-extendido.tex}
\input{capitulo-002-matematicas/teoria-de-numeros/aritmetica-modular/ecuaciones-diofanticas-lineales-de-dos.variables.tex}
\input{capitulo-002-matematicas/teoria-de-numeros/aritmetica-modular/conjunto-zn.tex}
\input{capitulo-002-matematicas/teoria-de-numeros/aritmetica-modular/propiedades-de-la-aritmetica-modular.tex}
\input{capitulo-002-matematicas/teoria-de-numeros/aritmetica-modular/inverzo-multiplicativo-en-zn.tex}


\section{combinatoria}
\subsection{principio multiplicativo}
si se quiere realizar un procedimiento de n pasos donde el primer paso puede ser hecho de $a_{1}$, el segundo paso
de $a_{2}$ y así sucesivamente hasta $a_{n}$ las formas de llevar a cabo el procedimiento son $a_{1}*a_{2} ... *a_{n}$

\subsection{numero de permutaciones de n elementos}
el numero de permutaciones es el numero de arreglos donde el orden importa, el numero de permutaciones
se calcula como $P(n,n)=n!$

\subsection{numero de permutaciones de n elementos tomados de a m}
el numero de permutaciones de n elementos tomados de a m son
$P(n,m)=\frac{n!}{\left (n-m  \right )!}$
\subsection{número de permutaciones de n elementos tomados de a m con repetición}
en este problema tenemos un suministro ilimitado de los n elementos diferentes y queremos saber de cuantas maneras podemos coger m elementos. su formula es:
$Pr(n,m)=n^{m}$

\subsection{Número de permutaciones con al menos un elemento fijo}
El número de permutaciones que tienen al menos un elemento fijo son todas las permutaciones que no son desarreglos\\
$n!-D_{n}$

\subsection{Número de permutaciones donde el primer elemento se repite a veces el segundo b veces ...}
el número de permutaciones es:
$\frac{n!}{a!b!...}$

\subsection{Número de desarreglos}
El numero de desarreglos es el numero de permutaciones que podemos hacer donde ninguno de los elementos esta en su posición inicial, se calculan con la siguiente formula recursiva.\\
$D_{n}=(n-1)(D_{n-1}+D_{n-2})$\\
casos base:
$D_{2}=1$ $D_{3}=2$
\subsection{Número de permutaciones de n elementos que dejan exactamente k elementos fijos}
El numero de permutaciones que dejan exactamente k elementos fijos es lo mismo que tachar k elementos y hacer un desarreglo con los n-k restantes. entonces la formula seria el numero de formas que podemos escoger k elementos del total multiplicado el desarreglo de n-k, siendo $s(n,k)$ el numero de arreglos con exactamente k elementos fijos tenemos:\\
$s(n,k)=c(n,k)D_{n-k}$

\subsection{número de combinaciones de n elementos tomados de a m}
el numero de combinaciones de n elementos cogidos de m son el numero de formas que podemos coger m elementos de los n sin importar su orden\\
\subsubsection{formas de calcular los números combinados o coeficientes binomiales}
formula:\\
$c(n,m)=\frac{n!}{m!(n-m)!}$\\
\\
formula recursiva:\\
casos bases $c(n,n)=c(n,0)=1$\\
de mas casos $c(n,m)=c(n-1,k)+c(n-1,k-1)$

\subsection{números figurados}
los números figurados, son números enteros  que son posibles representarlos como una figura geométrica, muchos de ellos tienen relación con la combinatoria
\subsubsection{números triangulares}
estos se pueden representar como un triangulo equilátero
\\\imagen{minipageSize=0.5\linewidth,height=6cm,caption=triangular.png}{capitulo-002-matematicas/combinatoria/imagenes/triangular.png}
\\son la suma de los primeros n números naturales y su relación con la combinatoria es la siguiente:
\\Los números triangulares se encuentran en el triangulo de pascal en la tercera fila del triangulo de pascal
\\\imagen{minipageSize=0.5\linewidth,height=6cm,caption=triangulares-pascal.png}{capitulo-002-matematicas/combinatoria/imagenes/triangulares-pascal.png}
\\y el triangulo de pascal lo podemos representar como números combinatorios de la siguiente forma:
\\\imagen{minipageSize=0.5\linewidth,height=6cm,caption=triangulo-de-pascal-combinatoria.png}{capitulo-002-matematicas/combinatoria/imagenes/triangulo-de-pascal-combinatoria.png}
\\Viendo en el triangulo de pascal podemos ver que podemos representar los números triangulares como la  $T_{n}=\binom{n+1}{2} $ o usando las propiedades de los números combinados como   $T_{n}=\binom{n+1}{n-1}$
\subsubsection{Números cuadráticos}
Los números cuadráticos se pueden representar como un cuadrado
\\\imagen{minipageSize=0.5\linewidth,height=6cm,caption=cuadraticos.png}{capitulo-002-matematicas/combinatoria/imagenes/cuadraticos.png}
\\tienen un propiedad  algo extraña pero fascinante un numero cuadrático en la suma de dos números triangulares continuos así que $n^{2} = T_{n} + T_{n-1}$

\subsection{Números de fibonacci}
Los números de fibonacci  ademas de aparecer en muchos de los patrones de la naturaleza también se pueden calcular con el triangulo de pascal
\\\imagen{minipageSize=0.5\linewidth,height=6cm,caption=fibonacci-pascal.png}{capitulo-002-matematicas/combinatoria/imagenes/fibonacci-pascal.png}
$fib(n+1)=\sum_{k=0}^{\frac{n}{2}}\binom{n-k}{k}$

\subsection{números de catalán}
los números de catalán son una secuencia de números naturales definidos como
\\$C_{n}={\frac {1}{n+1}}{2n \choose n}={\frac {(2n)!}{(n+1)!\,n!}}\qquad {\mbox{ con }}n\geq 0.$
como era de esperarse estos también pueden ser  calculados con el triangulo de pascal
\\\imagen{minipageSize=0.5\linewidth,height=6cm,caption=catalan-pascal.png}{capitulo-002-matematicas/combinatoria/imagenes/catalan-pascal.png}
\\y su fórmula con números combinatorios es $C_{n}={2n \choose n}-{2n \choose n-1}\quad {\mbox{ con }}n\geq 1.$
\subsubsection{Aplicaciones de los números de catalán}
\begin{itemize}
  \item son el número de expresiones que tienen n pares de paréntesis correctamente colocados, para n=3 tenemos ((()))	()(())	()()()	(())()	(()())
  \item son el número de formas de  partir un polígono convexo de n+2 lados en triángulos para n=2 tenemos
  \\\imagen{minipageSize=0.5\linewidth,height=6cm,caption=poligono-catalan.png}{capitulo-002-matematicas/combinatoria/imagenes/poligono-catalan.png}
  \item número de árboles binarios que se pueden construir que tenga n+1 hojas en los que cada nodo tiene 0 ó 2 hijos, para n=2 tenemos
  \\\imagen{minipageSize=0.5\linewidth,height=6cm,caption=arbol-catalan.png}{capitulo-002-matematicas/combinatoria/imagenes/arbol-catalan.png}
  \item número de caminos que se pueden trazar por las lineas de una cuadricula de n*n sin atravesar la diagonal, para n=2 tenemos
  \\\imagen{minipageSize=0.5\linewidth,height=6cm,caption=caminos-catalan.png}{capitulo-002-matematicas/combinatoria/imagenes/caminos-catalan.png}
\end{itemize}


\section{probabilidad}
\subsection{regla de Laplace}
la regla de laplace establece que la probabilidad de que ocurra un evento es la cantidad de casos favorables sobre la cantidad de casos posibles\\
$p(x)=\frac{casos\_favorables}{casos\_posibles}$

\subsection{probabilidad de intersección de sucesos}
si tenemos dos sucesos $a$ y $b$ la probabilidad de que suceda $a$ y $b$ es:
$p(a\cap b)=p(a)P(b|a)$\\

\subsection{Probabilidad de unión de sucesos}
Si tenemos dos sucesos $a$ y $b$ la probabilidad de que suceda $a$ ó $b$ es:
$p(a\cup b)=p(a)+p(b)-p(a\cap b)$

\subsection{Probabilidad condicionada}
La probabilidad condicionada es la probabilidad de que ocurra un evento $a$ sabiendo que ya ocurrió un evento $b$ y se calcula de la siguiente manera $p(a|b)=\frac{p(a\cap b)}{p(b)}$

\subsection{teorema de Bayes}
el teorema de Bayes indica una relación entre $p(a|b)$ y $p(b|a)$ y puede ser sacado de las formulas anteriores que hemos visto $p(a|b)=\frac{p(a)P(b|a)}{p(b)}$


\section{Potenciación rápida}
\subsection{Introducción}
La potenciación rápida es un algoritmo para calcular la potencia enésima de cualquier estructura donde este definida la multiplicación y el algoritmo es el siguiente:
\\\\mientras exponente sea diferente de 0 se repiten los siguientes pasos
\begin{itemize}
\item hacemos el resultado igual a la unidad, si el exponente es impar multiplicamos el resultado por nuestra base  osea $resultado = resultado * base$
\item hacemos la $base =  base * base$
\item tomamos la parte entera de dividir nuestro exponente por 2 $exponente=exponente/2$, estamos utilizando la propiedad de la potenciación que dice $\left ( 2^{n} \right )^m=2^{mn}$
\end{itemize}

\subsubsection{Ejemplo con un números enteros:}

\begin{table}[htbp]
\begin{center}
\begin{tabular}{|l|l|l|}
\hline
resultado & base & exponente \\
\hline \hline
1 &	2 &	13 \\ \hline
2 &	4 &	6 \\ \hline
2	& 16 & 3 \\ \hline
32 &	256 &	1 \\ \hline
8192 &	65536 &	0 \\ \hline
\end{tabular}
\caption{potenciacion rápida con números enteros.}
\label{tabla:ejemplo}
\end{center}
\end{table}
$2^{13}=2*\left ( 2^{2} \right )^{6}$
\\$2^{13}=2*\left ( 2^{4} \right )^3$
\\$2^{13}=2*2^{4}\left ( 2^{8} \right )^1$
\\$2^{13}=2*2^{4}*2^{8} \left ( 2^{16} \right )^0$


\subsubsection{Algoritmo general}
\begin{minipage}{\textwidth}
\begin{lstlisting}[style=C,caption=operadorPotencia]
Estructura operator^(const int& n) const
{
    Extructurra resultado(), base = *this;
    int exponente = n;
    while (exponente) {
        if (exponente & 1)//comprueba si exponente es impar
            resultado = resultado * base;
        exponente = exponente >> 1; //es lo mismo que exponente=exponente/2;
        base = base * base;
    }
    return resultado;
}
\end{lstlisting}
\end{minipage}


\section{transformaciones lineales}
una transformación lineal es una función que satisface los siguientes axiomas
\begin{itemize}
\item $f(x+y)=f(x)+f(y)$
\item $f(ax)=af(x)$ siendo a una constante
\end{itemize}


\chapter{Geometricos}
\section{formulas de geometría}
\begin{itemize}
\item $\frac{a}{sin(A)}=\frac{b}{sin(B)}=\frac{c}{sin(C)}$
\end{itemize}

\section{Estructuras geométricas}
\subsection{Puntos}
Un punto es una estructura matemática que no tiene dimensión, solo describe  una posición en el espacio. Pueden estar en el espacio
1d sobre una recta, 2d un plano … nd.
Sobre los puntos se pueden hacer varias operaciones que veremos mas adelante, la representación de un punto solo es un conjunto de
coordenadas que describen su posición, para una dimensión tendríamos un numero $x$, para dos dimensiones 2 números $x,y$
para 3 dimensiones $x,y,z$ y para n dimensiones tendríamos n números.
Estas son algunas de las formas de implementar en 2d un punto.
\begin{itemize}
\item Punto de enteros
\\
	\begin{lstlisting}[style=C]
	struct punto { int x, y;
	punto() { x = y = 0; }
	punto(int _x, int _y) : x(_x), y(_y) {} };
	\end{lstlisting}
	\item Punto de reales
	\\
	\begin{lstlisting}[style=C]
	struct punto { double x, y;
	punto() { x = y = 0.0; }
	point(double _x, double _y) : x(_x), y(_y) {} };
	\end{lstlisting}
\end{itemize}
\subsubsection{Operaciones con puntos}
\begin{itemize}
	\item Comparación
	 \\
	 Como algunos números son imposibles de representar en forma decimal por una computadora, las maquinas muchas veces aproximan el
	 resultado y esto da lugar inprecisiones  por ejemplo el numero $\frac{1}{3}$ no se puede representar en su totalidad por que tiene un número
	 de decimales infinitos, así que cuando estamos haciendo una comparación tenemos que comparar que  el valor absoluto de la resta de
	 2 valores es menor que $\varepsilon$, $\varepsilon$ es un numero muy pequeño casi cero se define normalmente como  1e-9.
	 \\
	 \begin{lstlisting}[style=C]
	 struct punto { double x, y;
	 punto() { x = y = 0.0; }
	 punto(double _x, double _y) : x(_x), y(_y) {}
	 bool operator == (punto otro) const {
	 return (fabs(x - otro.x) < EPS && (fabs(y - otro.y) < EPS));}};
	 \end{lstlisting}
	 \item Ordenamiento
	 \\
	 ordenar los puntos es muy importante en el caso de que estemos buscando optimizar la busqueda de cierto punto en un arreglo, para
	 que c++ pueda ordenar un arreglo la estructura debe tener definido el operador $<$ vamos a comprar por la coordenada $x$ y en caso
	 de empate compararemos la ordenada $y$
	 \begin{lstlisting}[style=C]
	 struct punto { double x, y;
	 punto() { x = y = 0.0; }
	 punto(double _x, double _y) : x(_x), y(_y) {}
	 bool operator < (punto otro) const {
	 if (fabs(x - otro.x) > EPS) return x < otro.x
	 return y < otro.y; } };
	 sort(P.begin(), P.end()); //ordenar existiendo el vector P
	 \end{lstlisting}
	 \item Distancia euclídea
	 \\
	 C++ tiene ya una función implementada para hallar la hipotenusa de un triangulo de rectángulo y es hypot y la usamos como muestra la imagen 3.1
	 \\\imagen{minipageSize=0.5\linewidth,height=6cm,caption=triangulo-cartesiano.png}{capitulo-004-geometricos/estructuras-geometricas/imagenes/triangulo-cartesiano.png}
	 \begin{lstlisting}[style=C]
	 double dist(punto p1, punto p2) {
	 return hypot(p1.x - p2.x, p1.y - p2.y);}
	 \end{lstlisting}
\end{itemize}

\subsection{Lineas}
\begin{lstlisting}[style=C]
struct line { double a, b, c; };
// a way to represent a line
\end{lstlisting}
\subsubsection{hallar una recta con 2 puntos}
\begin{lstlisting}[style=C]
// the answer is stored in the third parameter (pass by reference)
void pointsToLine(point p1, point p2, line &l) {
if (fabs(p1.x - p2.x) < EPS) {
// vertical line is fine
l.a = 1.0;
l.b = 0.0;
l.c = -p1.x;
// default values
} else {
l.a = -(double)(p1.y - p2.y) / (p1.x - p2.x);
l.b = 1.0;
// IMPORTANT: we fix the value of b to 1.0
l.c = -(double)(l.a * p1.x) - p1.y;
} }
\end{lstlisting}
\subsubsection{saber si dos lineas son paralelas}
\begin{lstlisting}[style=C]
bool areParallel(line l1, line l2) {
// check coefficients a & b
return (fabs(l1.a-l2.a) < EPS) && (fabs(l1.b-l2.b) < EPS); }
\end{lstlisting}
\subsubsection{saber si 2 lineas son iguales}
\begin{lstlisting}[style=C]
bool areSame(line l1, line l2) {
// also check coefficient c
return areParallel(l1 ,l2) && (fabs(l1.c - l2.c) < EPS); }
\end{lstlisting}
\subsubsection{saber si dos lineas son paralelas}
\begin{lstlisting}[style=C]
bool areParallel(line l1, line l2) {
// check coefficients a & b
return (fabs(l1.a-l2.a) < EPS) && (fabs(l1.b-l2.b) < EPS); }
\end{lstlisting}
\subsubsection{saber si 2 lineas son iguales}
\begin{lstlisting}[style=C]
bool areSame(line l1, line l2) {
// also check coefficient c
return areParallel(l1 ,l2) && (fabs(l1.c - l2.c) < EPS); }
\end{lstlisting}
\subsubsection{intersección entre 2 lineas}

\subsection{vectores}
Un vector es un segmento de linea que tiene magnitud y dirección, los vectores son representados parecido a como se
representa un punto con dos coordenadas $x, y$ donde con eso ya tenemos la magnitud y dirección del vector en posición estándar.
\begin{lstlisting}[style=C]
struct vec { double x, y;
vec(double _x, double _y) : x(_x), y(_y) {} };
\end{lstlisting}
Si tenemos un vector que no esta en posición estándar tenemos 2 puntos $cola$ y $cabeza$ donde para trasformarlo a posición estándar
solo tenemos que restar la cola con la cabeza.
\begin{lstlisting}[style=C]
vec vecAEstandar(punto cola, punto cabeza) {
return vec(cabeza.x - cola.x, cabez.y - cola.y); }
\end{lstlisting}
\subsubsection{Operaciones con vectores}
\begin{itemize}
  \item Escalar
  \\
  Es tener un vector con una magnitud igual a la que tenia multiplicado por un numero real positivo s con la misma dirección.
  \begin{lstlisting}[style=C]
  vec escalar(vec v, double s) {
    return vec(v.x * s, v.y * s);
  }
  \end{lstlisting}
  \item Cuadrado de la magnitud
  \\
  Como un vector es un segmento de linea su magnitud se puede hallar con la formula de la distancia euclídea, si
  no sacamos la raíz tenemos la magnitud al cuadrado
  \begin{lstlisting}[style=C]
  double cuadradoMagnitud(vec v) { return v.x * v.x + v.y * v.y; }
  \end{lstlisting}
  \item Producto punto
  \\
  El producto punto es una operación entre vectores donde el resultado es un escalar
  \begin{lstlisting}[style=C]
  double ProductoPunto(vec a, vec b) { return (a.x * b.x + a.y * b.y); }
  \end{lstlisting}
  \item Producto cruz
  \\
  Normal mente el producto cruz entre 2 vectores nos da otro vector, pero a nosotros solo nos interesa la magnitud por sus
  aplicaciones al plano 2d como el área del paralelogramo formado por 2 vectores. la magnitud del producto cruz la podemos aya de la
  siguiente manera
  \begin{lstlisting}[style=C]
  double cross(vec a, vec b) { return a.x * b.y - a.y * b.x; }
  \end{lstlisting}
\end{itemize}
\subsubsection{Aplicaciones de los vectores}
\begin{itemize}
  \item Área de un paralelogramo
  \\
  \\\imagen{minipageSize=0.5\linewidth,height=6cm,caption=paralelogramo.png}{capitulo-004-geometricos/estructuras-geometricas/imagenes/paralelogramo.png}
  \\
  $\left ( 2,5 \right )$ y $\left (4,1 \right )$
  \item Saber si un punto esta a la derecha o la izquierda  de una recta o esta dentro de la recta
  \\El producto punto se pude escribir también como $\sin \left ( \Theta  \right )\left | a \right |\left | b \right |$
  \\\imagen{minipageSize=0.5\linewidth,height=6cm,caption=dercha-izquierda-colineal.png}{capitulo-004-geometricos/estructuras-geometricas/imagenes/dercha-izquierda-colineal.png}
  \\
  Si el punto esta a la izquierda el seno del angulo sera positivo, si esta a la izquierda sera negativo y si es lineal sera 0.
\end{itemize}



\end{document}
